\documentclass[11pt]{aghdpl}
% \documentclass[en,11pt]{aghdpl}  % praca w języku angielskim

% Lista wszystkich języków stanowiących języki pozycji bibliograficznych użytych w pracy.
% (Zgodnie z zasadami tworzenia bibliografii każda pozycja powinna zostać utworzona zgodnie z zasadami języka, w którym dana publikacja została napisana.)
\usepackage[english,polish]{babel}

% Użyj polskiego łamania wyrazów (zamiast domyślnego angielskiego).
\usepackage{polski}

\usepackage[utf8]{inputenc}

% dodatkowe pakiety

\usepackage{mathtools}
\usepackage{amsfonts}
\usepackage{amsmath}
\usepackage{amsthm}

% --- < bibliografia > ---

\usepackage[
    style=numeric,
    sorting=none,
    %
    % Zastosuj styl wpisu bibliograficznego właściwy językowi publikacji.
    language=autobib,
    autolang=other,
    % Zapisuj datę dostępu do strony WWW w formacie RRRR-MM-DD.
    urldate=iso8601,
    % Nie dodawaj numerów stron, na których występuje cytowanie.
    backref=false,
    % Podawaj ISBN.
    isbn=true,
    % Nie podawaj URL-i, o ile nie jest to konieczne.
    url=false,
    %
    % Ustawienia związane z polskimi normami dla bibliografii.
    maxbibnames=3,
    % Jeżeli używamy BibTeXa:
    backend=bibtex
]{biblatex}

\usepackage{csquotes}
% Ponieważ `csquotes` nie posiada polskiego stylu, można skorzystać z mocno zbliżonego stylu chorwackiego.
\DeclareQuoteAlias{croatian}{polish}

\addbibresource{bibliografia.bib}

% Nie wyświetlaj wybranych pól.
%\AtEveryBibitem{\clearfield{note}}


% ------------------------
% --- < listingi > ---

% Użyj czcionki kroju Courier.
\usepackage{courier}

\usepackage{listings}
\lstloadlanguages{TeX}

\lstset{
    literate={ą}{{\k{a}}}1
    {ć}{{\'c}}1
    {ę}{{\k{e}}}1
    {ó}{{\'o}}1
    {ń}{{\'n}}1
    {ł}{{\l{}}}1
    {ś}{{\'s}}1
    {ź}{{\'z}}1
    {ż}{{\.z}}1
    {Ą}{{\k{A}}}1
    {Ć}{{\'C}}1
    {Ę}{{\k{E}}}1
    {Ó}{{\'O}}1
    {Ń}{{\'N}}1
    {Ł}{{\L{}}}1
    {Ś}{{\'S}}1
    {Ź}{{\'Z}}1
    {Ż}{{\.Z}}1,
    basicstyle=\footnotesize\ttfamily,
}

% ------------------------

\AtBeginDocument{
    \renewcommand{\tablename}{Tabela}
    \renewcommand{\figurename}{Rys.}
}

% ------------------------
% --- < tabele > ---

\usepackage{array}
\usepackage{tabularx}
\usepackage{multirow}
\usepackage{booktabs}
\usepackage{makecell}
\usepackage[flushleft]{threeparttable}

% defines the X column to use m (\parbox[c]) instead of p (`parbox[t]`)
\newcolumntype{C}[1]{>{\hsize=#1\hsize\centering\arraybackslash}X}


%---------------------------------------------------------------------------

\author{Kacper Żuk}
\shortauthor{K. Żuk}

\titlePL{Opracowanie biblioteki programistycznej do bezpiecznego uwierzytelniania urządzeń AVR.}
\titleEN{Development of libraries for authentication of AVR devices.}

\shorttitlePL{Opracowanie biblioteki programistycznej do bezpiecznego uwierzytelniania urządzeń AVR.}
\shorttitleEN{Development of libraries for authentication of AVR devices.}

\thesistype{Praca dyplomowa inżynierska}

\supervisor{dr inż. Jarosław Bułat}

\degreeprogramme{Teleinformatyka}

\date{2016}

\department{Katedra Telekomunikacji}

%\faculty{Wydział Elektrotechniki, Automatyki,\protect\\[-1mm] Informatyki i Inżynierii Biomedycznej}
\faculty{Wydział Informatyki, Elektroniki i Telekomunikacji}
%\faculty{Faculty of Electrical Engineering, Automatics, Computer Science and Biomedical Engineering}

% FIXME podziekowania?
\acknowledgements{}


\setlength{\cftsecnumwidth}{10mm}

%---------------------------------------------------------------------------
\setcounter{secnumdepth}{4}
\brokenpenalty=10000\relax

\begin{document}

\titlepages

% Ponowne zdefiniowanie stylu `plain`, aby usunąć numer strony z pierwszej strony spisu treści i poszczególnych rozdziałów.
\fancypagestyle{plain}
{
    % Usuń nagłówek i stopkę
    \fancyhf{}
    % Usuń linie.
    \renewcommand{\headrulewidth}{0pt}
    \renewcommand{\footrulewidth}{0pt}
}

\setcounter{tocdepth}{2}
\tableofcontents
\clearpage

\chapter{Wstęp}

O tym ze jest teraz IoT, o tym ze wazne jest bezpieczenstwo tego IoT (vide botnet Mirai), ze brak jest metod przystepnych dla userow, ze sa dostepne tylko prymitywy, ale brak zebrania tego wszystkiego razem w cos, w czym nie da sie popelnic bledu.

\chapter{Metody uwierzytelniania}
\section{Kryptografia asymetryczna}

RSA
ECC
ElGamal
DSA

Czym sie charakteryzuja, kto je rekomenduje i dlaczego.

https://www.keylength.com/en/4/
\section{Kryptografia symetryczna}

ECBC-MAC
OMAC
CCM
HMAC

\chapter{Charakterystyka platformy sprzętowej}

AVR, szeroko używane w IoT, jest fajne Arduino. 8-bitowy. Od ATtiny4 z 32B SRAM (http://www.atmel.com/Images/Atmel-8127-AVR-8-bit-Microcontroller-ATtiny4-ATtiny5-ATtiny9-ATtiny10_Datasheet-Summary.pdf) do ATxmega384C3 z 32KB SRAM (http://www.atmel.com/Images/Atmel-8361-8-and-16-bit-AVR-XMEGA-Microcontrollers-ATxmega384C3_Datasheet.pdf). 

\chapter{Implementacja}
\section{Generowanie współdzielonego klucza}
\section{Szyfrowanie wiadomości}
\section{Uwierzytelnienie wiadomości}
\section{Protokół komunikacji}

\chapter{Walidacja}

Informacja o bibliotece dla Javy i przykladowych implementacjach.

\chapter{Podsumowanie}

?


% itd.
% \appendix
% \chapter{Przykładowe wiadomości protokołu komunikacji}
\label{app:samplerecords}

\begin{table}[ht]
\centering
\caption{Budowa wiadomości typu HelloRequest}
\begin{BVerbatim}
# Nagłówek:
0x00 0x01 # wersja protokołu
0x00      # typ wiadomości: HelloRequest
0x00 0x40 # długość zawartości: 64 bajty

# Zawartość:
# 32 bajty współrzędnej X klucza publicznego
0x1F 0x92 0xDE 0xFF 0x6B 0xEE 0xFC 0x4D
0x51 0x2A 0x62 0xF4 0x60 0x1D 0x36 0x73
0x6F 0xEB 0x3F 0x1F 0x56 0x90 0xFD 0x85
0xB0 0x3C 0x56 0xD0 0xC0 0x52 0x6E 0x9B

# 32 bajty współrzędnej Y klucza publicznego
0xE8 0x60 0x84 0xB4 0xDE 0x73 0x65 0xB2
0x48 0xA6 0x15 0x79 0x7C 0xD9 0x4C 0xB6
0x56 0xE6 0xFA 0x3C 0x2F 0x3C 0x1C 0x8F
0xB2 0xE6 0x25 0xF8 0x66 0x2A 0x00 0xE4
\end{BVerbatim}
\label{fig:hellorequestsample}
\end{table}

\begin{table}
\centering
\caption{Budowa wiadomości typu HelloResponse}
\begin{BVerbatim}
# Nagłówek:
0x00 0x01 # wersja protokołu
0x01      # typ wiadomości: HelloResponse
0x00 0x60 # długość zawartości: 96 bajtów

# Zawartość:
# 16 bajtów kodu uwierzytelniającego
0x99 0x7B 0x57 0x10 0x7F 0x9C 0x1E 0xB3
0x1C 0x92 0x1B 0xFC 0x99 0x0C 0xCF 0x7D

# Zaszyfrowany klucz publiczny (64 bajty)
# z~uwzględnieniem dopełnienia PKCS#7 (16 bajtów)
0xF2 0x8A 0x7B 0xD4 0x04 0x80 0xAE 0xDC
0x7A 0x1D 0x04 0xEE 0x98 0x6D 0x9F 0xC5
0x22 0x4B 0x92 0x48 0x3E 0x65 0x79 0xC7
0xCB 0xCA 0xA9 0xC0 0xA1 0x7D 0x35 0x1F
0xFD 0x6F 0xE4 0x9E 0x62 0xBE 0x3F 0x1E
0xAC 0x32 0x03 0xF5 0x50 0x15 0xBE 0x0F
0x84 0xB9 0xF4 0xB6 0xF7 0x36 0x1D 0x3E
0x2D 0xD5 0xE3 0x10 0xBE 0xC4 0x39 0xC7
0x98 0x86 0x65 0x46 0x65 0xA9 0x5D 0xDE
0x4C 0x9D 0x03 0x00 0x55 0xE9 0xAF 0xC2
\end{BVerbatim}
\label{fig:helloresponsesample}
\end{table}

\begin{table}
\centering
\caption{Budowa wiadomości typu EncryptedData}
\begin{BVerbatim}
# Nagłówek:
0x00 0x01 # wersja protokołu
0x02      # typ wiadomości: EncryptedData
0x00 0x30 # długość zawartości: 48 bajtów

# Zawartość:
# 16 bajtów kodu uwierzytelniającego
0x6B 0xE7 0xBD 0xB6 0x23 0x98 0x1A 0x35
0xFF 0xBE 0xBB 0x38 0x3F 0xB1 0xF9 0x56

# Zaszyfrowane dane (dopełnione do~pełnego bloku AES)
0xD2 0x53 0xFF 0x58 0x99 0x94 0x36 0x24
0x1D 0x6B 0x4D 0x41 0x45 0xEF 0xD3 0x91
0x26 0x09 0xAB 0x91 0xC0 0x27 0x7C 0xAC
0x8E 0xFA 0xDA 0x24 0x9F 0xC6 0x22 0xBD
\end{BVerbatim}
\label{fig:encrypteddatasample}
\end{table}

% \chapter{Przykładowe bloki kodu biblioteki}
\label{app:codesamples}

\definecolor{mygreen}{rgb}{0,0.6,0}
\definecolor{mygray}{rgb}{0.5,0.5,0.5}
\definecolor{mymauve}{rgb}{0.58,0,0.82}

\lstset{%
  basicstyle=\footnotesize\ttfamily,%
  keywordstyle=\bfseries\color{green!40!black},%
  commentstyle=\itshape\color{purple!40!black},%
  identifierstyle=\color{blue},%
  stringstyle=\color{orange},%
  breakatwhitespace=false,         % sets if automatic breaks should only happen at whitespace
  breaklines=true,                 % sets automatic line breaking
  captionpos=t,                    % sets the caption-position to bottom
  frame=single,	                   % adds a frame around the code
  keepspaces=true,                 % keeps spaces in text, useful for keeping indentation of code (possibly needs columns=flexible)
  language=C,                 % the language of the code
  rulecolor=\color{black}         % if not set, the frame-color may be changed on line-breaks within not-black text (e.g. comments (green here))
}


\begin{figure}[h]
\begin{lstlisting}[caption={Szyfrowanie CBC wraz z obsługą dopełnienia PKCS\#7},label={lst:encrypt}]
size_t _seconn_crypto_encrypt(void *destination, void *source, size_t length, aes128_key_t enc_key) {
    uint8_t *dest = (uint8_t*)destination;
    uint8_t *src = (uint8_t*)source;

    rng(dest, 16); // losowy wektor inicjalizacyjny
    memset(&ctx, 0, sizeof(aes128_ctx_t));

    aes128_init(enc_key, &ctx);

    aes128_enc(dest, &ctx);

    size_t i = 0;
    for(; i+16 <= length; i += 16) {
        memcpy(dest+16+i, src+i, 16);
        _seconn_crypto_xor_block(dest+16+i, dest+i);
        aes128_enc(dest+16+i, &ctx);
    }

    size_t pad_length = 16 - (length % 16);
    memset(dest+16+i, pad_length, 16);
    memcpy(dest+16+i, src+i, length - i);
    _seconn_crypto_xor_block(dest+16+i, dest+i);
    aes128_enc(dest+16+i, &ctx);

    return i+32;
}
\end{lstlisting}
\end{figure}

\begin{figure}
\begin{lstlisting}[caption={Odszyfrowanie CBC wraz z obsługą dopełnienia PKCS\#7},label={lst:decrypt}]
size_t _seconn_crypto_decrypt(void *destination, void *source, size_t length, aes128_key_t enc_key) {
    uint8_t *src = ((uint8_t*)source);
    uint8_t *dest = (uint8_t*)destination;

    memset(&ctx, 0, sizeof(aes128_ctx_t));
    aes128_init(enc_key, &ctx);

    size_t i = 0;
    for(; i+16 < length; i += 16) {
        memcpy(dest+i, src+i+16, 16);
        aes128_dec(dest+i, &ctx);
        _seconn_crypto_xor_block(dest+i, src+i);
    }

    size_t pad_length = dest[i-1];
    for(size_t j = 2; j <= pad_length; j++) {
        if (dest[i-j] != pad_length) {
            return 0;
        }
    }

    return i-pad_length;
}
\end{lstlisting}
\end{figure}

\begin{figure}
\begin{lstlisting}[caption={Obliczanie \gls{mac} dla wiadomości},label={lst:mac}]
void _seconn_crypto_calculate_mac(uint8_t *mac, void *message,
size_t length, aes128_key_t mac_key, aes128_key_t enc_key) {
    memset(&ctx, 0, sizeof(aes128_ctx_t));
    aes128_init(mac_key, &ctx);

    uint8_t *block = mac;

    memset(block, 0, 16); // wektor inicjalizacyjny wypełniony zerami

    // obliczenie ostatniego bloku szyfrowania w trybie CBC
    size_t i = 0;
    for(; i+16 <= length; i += 16) {
        _seconn_crypto_xor_block(block, ((uint8_t*)message)+i);
        aes128_enc(block, &ctx);
    }

    // szyfrowanie ostatniego bloku osobnym kluczem
    memset(&ctx, 0, sizeof(aes128_ctx_t));
    aes128_init(enc_key, &ctx);
    aes128_enc(block, &ctx);
}
\end{lstlisting}
\end{figure}

% itd.

\printbibliography

\end{document}
