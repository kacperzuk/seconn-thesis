\documentclass[oneside,11pt]{aghdpl}

% \documentclass[oneside,11pt]{aghdpl}
% \documentclass[en,11pt]{aghdpl}  % praca w języku angielskim

% Lista wszystkich języków stanowiących języki pozycji bibliograficznych użytych w pracy.
% (Zgodnie z zasadami tworzenia bibliografii każda pozycja powinna zostać utworzona zgodnie z zasadami języka, w którym dana publikacja została napisana.)
\usepackage[english,polish]{babel}

% Użyj polskiego łamania wyrazów (zamiast domyślnego angielskiego).
\usepackage{polski}

\usepackage[utf8]{inputenc}

% dodatkowe pakiety

\usepackage{mathtools}
\usepackage{amsfonts}
\usepackage{amsmath}
\usepackage{amsthm}
\usepackage{fancyvrb}
\usepackage[acronyms]{glossaries}
\makeglossaries
\loadglsentries[main]{acronyms}

% --- < bibliografia > ---

\usepackage[
    style=numeric,
    sorting=none,
    %
    % Zastosuj styl wpisu bibliograficznego właściwy językowi publikacji.
    language=autobib,
    autolang=other,
    % Zapisuj datę dostępu do strony WWW w formacie RRRR-MM-DD.
    urldate=iso8601,
    % Nie dodawaj numerów stron, na których występuje cytowanie.
    backref=false,
    % Podawaj ISBN.
    isbn=true,
    % Nie podawaj URL-i, o ile nie jest to konieczne.
    url=false,
    %
    % Ustawienia związane z polskimi normami dla bibliografii.
    maxbibnames=3,
    % Jeżeli używamy BibTeXa:
    backend=bibtex
]{biblatex}

\usepackage{csquotes}
% Ponieważ `csquotes` nie posiada polskiego stylu, można skorzystać z mocno zbliżonego stylu chorwackiego.
\DeclareQuoteAlias{croatian}{polish}

\addbibresource{bibliografia.bib}

% Nie wyświetlaj wybranych pól.
%\AtEveryBibitem{\clearfield{note}}


% ------------------------
% --- < listingi > ---

% Użyj czcionki kroju Courier.
\usepackage{courier}

\usepackage{listings}
\lstloadlanguages{TeX}

\lstset{
    literate={ą}{{\k{a}}}1
    {ć}{{\'c}}1
    {ę}{{\k{e}}}1
    {ó}{{\'o}}1
    {ń}{{\'n}}1
    {ł}{{\l{}}}1
    {ś}{{\'s}}1
    {ź}{{\'z}}1
    {ż}{{\.z}}1
    {Ą}{{\k{A}}}1
    {Ć}{{\'C}}1
    {Ę}{{\k{E}}}1
    {Ó}{{\'O}}1
    {Ń}{{\'N}}1
    {Ł}{{\L{}}}1
    {Ś}{{\'S}}1
    {Ź}{{\'Z}}1
    {Ż}{{\.Z}}1,
    basicstyle=\footnotesize\ttfamily,
}

% ------------------------

\AtBeginDocument{
    \renewcommand{\tablename}{Tabela}
    \renewcommand{\figurename}{Rys.}
}

% ------------------------
% --- < tabele > ---

\usepackage{array}
\usepackage{tabularx}
\usepackage{multirow}
\usepackage{booktabs}
\usepackage{makecell}
\usepackage[flushleft]{threeparttable}

% defines the X column to use m (\parbox[c]) instead of p (`parbox[t]`)
\newcolumntype{C}[1]{>{\hsize=#1\hsize\centering\arraybackslash}X}

\DefineBibliographyStrings{english}{%
  urlseen = {dostęp dnia},
}

%---------------------------------------------------------------------------

\author{Kacper Żuk}
\shortauthor{K. Żuk}

\titlePL{Opracowanie biblioteki programistycznej do bezpiecznego uwierzytelniania urządzeń AVR.}
\titleEN{Development of libraries for authentication of AVR devices.}

\shorttitlePL{Opracowanie biblioteki programistycznej do bezpiecznego uwierzytelniania urządzeń AVR.}
\shorttitleEN{Development of libraries for authentication of AVR devices.}

\thesistype{Praca dyplomowa inżynierska}

\supervisor{dr inż. Jarosław Bułat}

\degreeprogramme{Teleinformatyka}

\date{2016}

\department{Katedra Telekomunikacji}

%\faculty{Wydział Elektrotechniki, Automatyki,\protect\\[-1mm] Informatyki i Inżynierii Biomedycznej}
\faculty{Wydział Informatyki, Elektroniki i Telekomunikacji}
%\faculty{Faculty of Electrical Engineering, Automatics, Computer Science and Biomedical Engineering}

% FIXME podziekowania?
\acknowledgements{}


\setlength{\cftsecnumwidth}{10mm}

%---------------------------------------------------------------------------
\setcounter{secnumdepth}{4}
\brokenpenalty=10000\relax

\begin{document}

\titlepages

% Ponowne zdefiniowanie stylu `plain`, aby usunąć numer strony z pierwszej strony spisu treści i poszczególnych rozdziałów.
\fancypagestyle{plain}
{
    % Usuń nagłówek i stopkę
    \fancyhf{}
    % Usuń linie.
    \renewcommand{\headrulewidth}{0pt}
    \renewcommand{\footrulewidth}{0pt}
}

\setcounter{tocdepth}{2}
\tableofcontents
\clearpage

\glsaddall
\printglossary[type=\acronymtype, style=super, nonumberlist, title=Spis skrótów]

\chapter{Wstęp}
\label{cha:wstep}

O tym ze jest teraz IoT, o tym ze wazne jest bezpieczenstwo tego IoT (vide botnet Mirai), ze brak jest metod przystepnych dla userow, ze sa dostepne tylko prymitywy, ale brak zebrania tego wszystkiego razem w cos, w czym nie da sie popelnic bledu.

\chapter{Charakterystyka platformy sprzętowej}
\label{cha:hardware}

Mikropocesory Atmel AVR są w większości 8-bitowe i na takich skupia się ta praca. Rodzina AVR jest dość szeroka, od ATtiny4 z 32B SRAM~\cite{Attiny4} do ATxmega384C3 z 32KB SRAM~\cite{Atxmega384}. 

\chapter{Metody uwierzytelniania}
\label{cha:metodyUwierzytelniania}

W zależności od potrzeb i ograniczeń stosuje się różne metody uwierzytelniania podmiotów w komunikacji. Wyróżnić należy uwierzytelnianie przy pomocy kryptografii asymetrycznej, w której używana jest para matematycznie związanych ze sobą kluczy, oraz uwierzytelnianie przy pomocy kryptografii symetrycznej, w której używany jest jeden, współdzielony, tajny klucz.

Klucze w przypadku kryptografii asymetrycznej muszą posiadać konkretne właściwości. W przypadku RSA bezpieczeństwo polega na trudności w faktoryzowaniu dużych liczb, co wymaga stosowania kluczy co najmniej 2048 bitowych~\cite{Nist}. Klucze w przypadku kryptografii symetrycznej nie muszą mieć konkretnych właściwości poza ich nieprzewidywalnością.

Ważną różnicą jest też wydajność. Kryptografia asymetryczna jest dużo bardziej złożona obliczeniowo od symetrycznej~\cite{al2008comparative}. Jest to szczególnie istotne na ograniczonych sprzętowo systemach wbudowanych. Przewagą kryptografii asymetrycznej jest jednak brak konieczności ustalenia wspólnego klucza przed rozpoczęciem komunikacji, jak ma to miejsce w przypadku kryptografii symetrycznej.

Zalecanym rozwiązaniem jest najpierw ustalenie wspólnego, tajnego klucza przy użyciu kryptografii asymetrycznej, a następnie użycie tego klucza do kryptografii symetrycznej~\cite{al2008comparative}.

\section{Kryptografia asymetryczna}
\label{sec:kryptoAsym}

Przy wyborze algorytmu dla potrzeb pracy istotne były:

\begin{itemize}
\item jakość implementacji dostępnych na mikroprocesory AVR,
\item złożoność obliczeniowa (niższa jest lepsza),
\item długość klucza wymagana do zapewnienia niezbędnego poziomu bezpieczeństwa.
\end{itemize}

Biblioteka \emph{AVR-Crypto-Lib} dostarcza implementację algorytmów RSA oraz DSA\footnote{\url{https://trac.cryptolib.org/avr-crypto-lib/browser}}. Biblioteka \emph{Emsign} dostarcza implementację RSA, lecz tylko z 64 bitowym kluczem\footnote{\url{http://www.emsign.nl/}}, co nie jest wystarczające dla zapewnienia bezpieczeństwa. Komercyjna biblioteka \emph{LightCrypt-AVR8-ECC} oraz biblioteka \emph{micro-ecc} dostarczają implementację kryptografii opartej o krzywe eliptyczne\footnote{\url{http://industrial.crypto.cmmsigma.eu/lightcrypt_avr8/lc_avr8_ecc.pl.html}}. Brak jest na rynku implementacji innych algorytmów klucza publicznego. Dostępność implementacji ogranicza wybór algorytmu do RSA, DSA oraz krzywych eliptycznych.

Następnym kryterium jest złożoność obliczeniowa. W analizie przeprowadzonej przez pracowników \emph{Sun Microsystems Laboratories} wykazano, że na mikroprocesorach AVR algorytmy oparte o krzywe eliptyczne są o rząd wielkości szybsze od algorytmu RSA~\cite{Gura2004}.

FIXME
DSA jest podobne do RSA, wiec tez nie

FIXME
dlaczego taki a nie inny secp
https://www.keylength.com/en/4/

FIXME
ECDH jest fajne

FIXME
ale nie daje forward secrecy



\section{Kryptografia symetryczna}
\label{sec:kryptoSym}

ECBC-MAC
OMAC
CCM
HMAC

\chapter{Implementacja protokołu komunikacji}
\label{cha:implementacja}

Protokół komunikacji między dwoma węzłami zaprojektowano i zaimplementowano z następującymi założeniami:

\begin{itemize}
\item pełna funkcjonalność przy jak najmniejszych wymaganiach sprzętowych
\item niezależność od warstwy sieciowej
\item zapewnienie uwierzytelniania i szyfrowania wiadomości
\end{itemize}

\section{Podstawowe struktury protokołu}
\label{sec:proto}

W protokole wymieniane są rekordy, których zawartość poprzedzona jest nagłówkiem o następującej budowie:

\begin{itemize}
\item piersze dwa bajty definiują wersję protokołu i mają wartość 0x00 0x01,
\item następny bajt definiuje typ rekordu,
\item następne dwa bajty definiują długość zawartości rekordu (najbardziej znaczący bajt jako pierwszy).
\end{itemize}

\begin{table}[t]
\centering
\begin{tabular}{|l|p{1.4cm}|l|p{2.9cm}|p{3.1cm}|}
    \hline
    \textbf{Nazwa rekordu}  &
    \textbf{Wartość pola typ}  &
    \textbf{Długość zawartości}  &
    \textbf{Czy zawartość jest zaszyfrowana}  &
    \textbf{Czy zawartość jest uwierzytelniona}\\
    \hline
    HelloRequest & 0x00 & 64 bajty & nie & nie\\
    \hline
    HelloResponse & 0x01 & 96 bajtów & tak & tak\\
    \hline
    EncryptedData & 0x02 & zmienna, minimum 32 bajty & tak & tak\\
    \hline
\end{tabular}
\caption{Typy rekordów wraz z ich charakterystyką}
\label{tab:recordtypes}
\end{table}


Zdefiniowane typy rekordów zostały przedstawione w tabeli \ref{tab:recordtypes}.

Budowę przykładowych rekordów przedstawiono na rysunkach \ref{fig:hellorequestsample}, \ref{fig:helloresponsesample} oraz \ref{fig:encrypteddatasample} (Załącznik \ref{app:samplerecords}).

Odbiorca rekordu powinien zweryfikować:

\begin{itemize}
\item zgodność wersji protokołu -- wymagane bajty 0x00 oraz 0x01,
\item prawidłowość bajtu określającego typ rekordu -- wymagana wartość 0x00, 0x01 lub 0x02,
\item zgodność zadeklarowanej długości zawartości rekordu z typem rekordu,
\item w przypadku typów HelloResponse oraz EncryptedData -- prawidłowość kodu uwierzytelniającego.
\end{itemize}

W przypadku niezgodności któregokolwiek elementu rekord powinien zostać zignorowany.

\section{Nawiązywanie połączenia}

\begin{figure}
\centering
\begin{BVerbatim}
Nawiązujący połączenie              Odbierający połączenie
        +                                       +
        |   HelloRequest                        |
   1.   | +-----------------------------------> |
        |                                       |
        |                                       |
        |                        HelloRequest   |
   2.   | <-----------------------------------+ |
        |                                       |
        |                                       |
        |   HelloResponse                       |
   3.   | +-----------------------------------> |
        |                                       |
        |                                       |
        |                       HelloResponse   |
   4.   | <-----------------------------------+ |
        +                                       +
\end{BVerbatim}
\caption{Kolejność wymiany rekordów w procesie nawiązywania połączenia}
\label{fig:handshake}
\end{figure}

Kolejność przesyłania rekordów w celu nawiązania połączenia przedstawiona jest na rysunku~\ref{fig:handshake}. Rekordy HelloRequest zawierają klucz publiczny węzła, który je wysyła. Węzeł, który odbiera HelloRequest używa swojego klucza publicznego oraz klucza publicznego z odebranego rekordu do ustalenia sekretnego klucza.

Po ustaleniu wspólnego klucza węzły mogą wysłać rekord HelloResponse, który zawiera zaszyfrowany i uwierzytelniony klucz publiczny węzła wysyłającego rekord. Jeżeli węzeł odbierający rekord jest w stanie potwierdzić, że rekord jest prawidłowo uwierzytelniony, a zdeszyfrowany klucz publiczny pokrywa się z kluczem przesłanym wcześniej w rekordzie HelloRequest, połączenie uznawane jest za nawiązane. Po nawiązaniu połączenia wymieniane mogą być tylko rekordy typu EncryptedData.

\section{Generowanie współdzielonego klucza}
\label{sec:sharedkey}

Każdy z węzłów po odebraniu rekordu HelloRequest używa odebranego klucza publicznego oraz swojego klucza publicznego do ustalenia wspólnego sekretu przy użyciu algorytmu \gls{ecdh} oraz proponowanej przez \gls{nist} krzywej eliptycznej P-256~\cite{kerry2013digital} (w RFC 5480 nazwaną krzywą secp256r1~\cite{turner2009elliptic}).

Z sekretu będącego wynikiem algorytmu \gls{ecdh} liczony jest skrót przy użyciu algorytmu SHA-256. Następnie jest on dzielony na dwie części po 128-bitów. Pierwsza część staje się współdzielonym kluczem używanym do szyfrowania, druga część staje się współdzielonym kluczem używanym do uwierzytelniania.

Implementacja algorytmu \gls{ecdh} z krzywą eliptyczną P-256 pochodzi z biblioteki \emph{micro-ecc}. Jest to jedyna darmowa biblioteka implementująca \gls{ecdh}.

\section{Szyfrowanie i deszyfrowanie wiadomości}
\label{sec:encrypt}

Szyfrowanie wiadomości odbywa się za pomocą szyfru blokowego AES ze 128-bitowym kluczem używanym w trybie \gls{cbc}. Wektor inicjalizacyjny jest losowy i dołączany przed szyfrogramem. Tekst jawny jest dopełniany do pełnego bloku według algorytmu zdefiniowanego w PKCS\#7~\cite{kaliski1998pkcs}.

Właściwe kroki potrzebne do zaszyfrowania wiadomości:

\begin{enumerate}
\item dopełnienie tekstu jawnego do pełnego bloku
\begin{itemize}
\item jeżeli długość tekstu jawnego jest wielokrotnością długości bloku, do tekstu jawnego doklejone musi być 128 bajtów o wartości 128.
\item w przeciwnym wypadku, gdy wymagane jest dopełnienie $ N $ bajtów, do tekstu jawnego doklejone musi być $ N $ bajtów o wartości $ N $.
\end{itemize}
\item zaszyfrowanie dopełnionego tekstu jawnego w trybie \gls{cbc} z losowym wektorem inicjalizacyjnym
\item doklejenie przed szyfrogramem wektora inicjalizacyjnego
\end{enumerate}

Właściwe kroki potrzebne do odszyfrowania wiadomości:

\begin{enumerate}
\item oddzielenie wektora inicjalizacyjnego od szyfrogramu
\item zdeszyfrowanie szyfrogramu w trybu \gls{cbc} przy wykorzystaniu oddzielonego wektora inicjalizacyjnego
\item pobranie wartości ostatniego bajtu zdeszyfrowanego ciągu
\begin{itemize}
    \item wartość ta nazywana jest dalej $ N $
\end{itemize}
\item zweryfikowanie poprawności dopełnienia
\begin{itemize}
\item ostatnie $ N $ bajtów musi mieć wartość $ N $
\item jeżeli dopełnienie jest nieprawidłowe, cały rekord jest ignorowany
\end{itemize}
\item usunięcie ostatnich $ N $ bajtów
\end{enumerate}

Implementacja algorytmu AES pochodzi z biblioteki \emph{AVR-Crypto-Lib}. Jest to najlepiej udokumentowana, darmowa biblioteka implementująca algorytm AES. Implementacja trybu \gls{cbc} oraz algorytmu dopełniania została zrealizowana w ramach pracy.

\section{Uwierzytelnienie wiadomości}
\label{sec:auth}

Szyfrowanie wiadomości odbywa się za pomocą szyfru blokowego AES ze 128-bitowym kluczem używanym w trybie \gls{ecbcmac}. Wektor inicjalizacyjny wypełniony jest zerami i nie jest przesyłany. Uwierzytelniany jest kompletny szyfrogram wraz z wektorem inicjalizacyjnym użytym do szyfrowania, a nie tekst jawny. Długość szyfrogramu wraz z wektorem inicjalizacyjnym zawsze będzie wielokrotnością długości bloku, a więc nie jest stosowane dopełnianie.

Tryb \gls{ecbcmac} to tryb \gls{cbcmac}, którego wynik jest dodatkowo szyfrowany innym kluczem niż ten użyty do \gls{cbcmac}. W tej pracy do \gls{cbcmac} użyty jest klucz służący do uwierzytelniania, a wynik \gls{cbcmac} jest zaszyfrowany używając klucza służącego do szyfrowania.

Właściwe kroki potrzebne do obliczenia kodu uwierzytelniającego:

\begin{enumerate}
\item obliczenie ostatniego bloku będącego wynikiem zaszyfrowania szyfrogramu w trybie \gls{cbc} z wektorem inicjalizacyjnym wypełnionym zerami przy użyciu klucza przeznaczonego do uwierzytelniania
\item zaszyfrowanie bloku przy wykorzystaniu AES i klucza przeznaczonego do szyfrowania
\end{enumerate}

Węzeł wysyłający dokleja kod uwierzytelniający przed szyfrogramem. Węzeł odbierający oddziela otrzymany kod od szyfrogramu, oblicza kod uwierzytelniający dla danego szyfrogramu i porównuje, czy zgadza się on z kodem otrzymanym. Jeżeli kod obliczony różni się od kodu otrzymanego, cały rekord jest ignorowany.

Implementacja trybu \gls{ecbcmac} została zrealizowana w ramach pracy.

\chapter{Walidacja}
\label{cha:walidacja}

Walidacji poddane zostały trzy aspekty:

\begin{itemize}
\item poprawność zaprojektowania interfejsu programistycznego stworzonej biblioteki
\item możliwość poprawnego nawiązania połączenia i przesłania danych między dwoma węzłami
\item poprawność implementacji algorytmów kryptograficznych
\end{itemize}

\section{Poprawność interfejsu programistycznego}

Poprawny interfejs programistyczny biblioteki musi umożliwiać stworzenie pełnego rozwiązania zapewniającego bezpieczną komunikację. Zostało to zweryfikowane poprzez stworzenie przykładowego oprogramowania wykorzystującego bibliotekę. Oprogramowanie to powstało na platformę Arduino oraz wykorzystuje moduł bluetooth XM-15B.

Po uruchomieniu urządzenia moduł bluetooth zostanie skonfigurowany w trybie \emph{slave} i 

\section{Poprawność komunikacji}

\section{Poprawność implementacji algorytmów kryptograficznych}

Poprawność implementacji algorytmów \gls{cbc} i dopełniania według PKCS\#7 stworzonych w ramach pracy oraz algorytmów AES, SHA-256 oraz ECDH dostarczonych przez zewnętrzne biblioteki została zwalidowana poprzez stworzenie drugiej implementacji protokołu w języku Java.

Implementacje algorytmów AES, CBC, dopełniania według PKCS\#7, SHA-256 oraz ECDH pochodzą z pakietu java.security biblioteki standardowej języka Java. Implementacja ECBC-MAC została wykonana w ramach pracy w oparciu o implementację CBC dostarczoną przez bibliotekę standardową.

\chapter{Podsumowanie}
\label{cha:podsumowanie}

?


% itd.
% \appendix
% \chapter{Przykładowe wiadomości protokołu komunikacji}
\label{app:samplerecords}

\begin{table}[ht]
\centering
\caption{Budowa wiadomości typu HelloRequest}
\begin{BVerbatim}
# Nagłówek:
0x00 0x01 # wersja protokołu
0x00      # typ wiadomości: HelloRequest
0x00 0x40 # długość zawartości: 64 bajty

# Zawartość:
# 32 bajty współrzędnej X klucza publicznego
0x1F 0x92 0xDE 0xFF 0x6B 0xEE 0xFC 0x4D
0x51 0x2A 0x62 0xF4 0x60 0x1D 0x36 0x73
0x6F 0xEB 0x3F 0x1F 0x56 0x90 0xFD 0x85
0xB0 0x3C 0x56 0xD0 0xC0 0x52 0x6E 0x9B

# 32 bajty współrzędnej Y klucza publicznego
0xE8 0x60 0x84 0xB4 0xDE 0x73 0x65 0xB2
0x48 0xA6 0x15 0x79 0x7C 0xD9 0x4C 0xB6
0x56 0xE6 0xFA 0x3C 0x2F 0x3C 0x1C 0x8F
0xB2 0xE6 0x25 0xF8 0x66 0x2A 0x00 0xE4
\end{BVerbatim}
\label{fig:hellorequestsample}
\end{table}

\begin{table}
\centering
\caption{Budowa wiadomości typu HelloResponse}
\begin{BVerbatim}
# Nagłówek:
0x00 0x01 # wersja protokołu
0x01      # typ wiadomości: HelloResponse
0x00 0x60 # długość zawartości: 96 bajtów

# Zawartość:
# 16 bajtów kodu uwierzytelniającego
0x99 0x7B 0x57 0x10 0x7F 0x9C 0x1E 0xB3
0x1C 0x92 0x1B 0xFC 0x99 0x0C 0xCF 0x7D

# Zaszyfrowany klucz publiczny (64 bajty)
# z~uwzględnieniem dopełnienia PKCS#7 (16 bajtów)
0xF2 0x8A 0x7B 0xD4 0x04 0x80 0xAE 0xDC
0x7A 0x1D 0x04 0xEE 0x98 0x6D 0x9F 0xC5
0x22 0x4B 0x92 0x48 0x3E 0x65 0x79 0xC7
0xCB 0xCA 0xA9 0xC0 0xA1 0x7D 0x35 0x1F
0xFD 0x6F 0xE4 0x9E 0x62 0xBE 0x3F 0x1E
0xAC 0x32 0x03 0xF5 0x50 0x15 0xBE 0x0F
0x84 0xB9 0xF4 0xB6 0xF7 0x36 0x1D 0x3E
0x2D 0xD5 0xE3 0x10 0xBE 0xC4 0x39 0xC7
0x98 0x86 0x65 0x46 0x65 0xA9 0x5D 0xDE
0x4C 0x9D 0x03 0x00 0x55 0xE9 0xAF 0xC2
\end{BVerbatim}
\label{fig:helloresponsesample}
\end{table}

\begin{table}
\centering
\caption{Budowa wiadomości typu EncryptedData}
\begin{BVerbatim}
# Nagłówek:
0x00 0x01 # wersja protokołu
0x02      # typ wiadomości: EncryptedData
0x00 0x30 # długość zawartości: 48 bajtów

# Zawartość:
# 16 bajtów kodu uwierzytelniającego
0x6B 0xE7 0xBD 0xB6 0x23 0x98 0x1A 0x35
0xFF 0xBE 0xBB 0x38 0x3F 0xB1 0xF9 0x56

# Zaszyfrowane dane (dopełnione do~pełnego bloku AES)
0xD2 0x53 0xFF 0x58 0x99 0x94 0x36 0x24
0x1D 0x6B 0x4D 0x41 0x45 0xEF 0xD3 0x91
0x26 0x09 0xAB 0x91 0xC0 0x27 0x7C 0xAC
0x8E 0xFA 0xDA 0x24 0x9F 0xC6 0x22 0xBD
\end{BVerbatim}
\label{fig:encrypteddatasample}
\end{table}

% \chapter{Przykładowe bloki kodu biblioteki}
\label{app:codesamples}

\definecolor{mygreen}{rgb}{0,0.6,0}
\definecolor{mygray}{rgb}{0.5,0.5,0.5}
\definecolor{mymauve}{rgb}{0.58,0,0.82}

\lstset{%
  basicstyle=\footnotesize\ttfamily,%
  keywordstyle=\bfseries\color{green!40!black},%
  commentstyle=\itshape\color{purple!40!black},%
  identifierstyle=\color{blue},%
  stringstyle=\color{orange},%
  breakatwhitespace=false,         % sets if automatic breaks should only happen at whitespace
  breaklines=true,                 % sets automatic line breaking
  captionpos=t,                    % sets the caption-position to bottom
  frame=single,	                   % adds a frame around the code
  keepspaces=true,                 % keeps spaces in text, useful for keeping indentation of code (possibly needs columns=flexible)
  language=C,                 % the language of the code
  rulecolor=\color{black}         % if not set, the frame-color may be changed on line-breaks within not-black text (e.g. comments (green here))
}


\begin{figure}[h]
\begin{lstlisting}[caption={Szyfrowanie CBC wraz z obsługą dopełnienia PKCS\#7},label={lst:encrypt}]
size_t _seconn_crypto_encrypt(void *destination, void *source, size_t length, aes128_key_t enc_key) {
    uint8_t *dest = (uint8_t*)destination;
    uint8_t *src = (uint8_t*)source;

    rng(dest, 16); // losowy wektor inicjalizacyjny
    memset(&ctx, 0, sizeof(aes128_ctx_t));

    aes128_init(enc_key, &ctx);

    aes128_enc(dest, &ctx);

    size_t i = 0;
    for(; i+16 <= length; i += 16) {
        memcpy(dest+16+i, src+i, 16);
        _seconn_crypto_xor_block(dest+16+i, dest+i);
        aes128_enc(dest+16+i, &ctx);
    }

    size_t pad_length = 16 - (length % 16);
    memset(dest+16+i, pad_length, 16);
    memcpy(dest+16+i, src+i, length - i);
    _seconn_crypto_xor_block(dest+16+i, dest+i);
    aes128_enc(dest+16+i, &ctx);

    return i+32;
}
\end{lstlisting}
\end{figure}

\begin{figure}
\begin{lstlisting}[caption={Odszyfrowanie CBC wraz z obsługą dopełnienia PKCS\#7},label={lst:decrypt}]
size_t _seconn_crypto_decrypt(void *destination, void *source, size_t length, aes128_key_t enc_key) {
    uint8_t *src = ((uint8_t*)source);
    uint8_t *dest = (uint8_t*)destination;

    memset(&ctx, 0, sizeof(aes128_ctx_t));
    aes128_init(enc_key, &ctx);

    size_t i = 0;
    for(; i+16 < length; i += 16) {
        memcpy(dest+i, src+i+16, 16);
        aes128_dec(dest+i, &ctx);
        _seconn_crypto_xor_block(dest+i, src+i);
    }

    size_t pad_length = dest[i-1];
    for(size_t j = 2; j <= pad_length; j++) {
        if (dest[i-j] != pad_length) {
            return 0;
        }
    }

    return i-pad_length;
}
\end{lstlisting}
\end{figure}

\begin{figure}
\begin{lstlisting}[caption={Obliczanie \gls{mac} dla wiadomości},label={lst:mac}]
void _seconn_crypto_calculate_mac(uint8_t *mac, void *message,
size_t length, aes128_key_t mac_key, aes128_key_t enc_key) {
    memset(&ctx, 0, sizeof(aes128_ctx_t));
    aes128_init(mac_key, &ctx);

    uint8_t *block = mac;

    memset(block, 0, 16); // wektor inicjalizacyjny wypełniony zerami

    // obliczenie ostatniego bloku szyfrowania w trybie CBC
    size_t i = 0;
    for(; i+16 <= length; i += 16) {
        _seconn_crypto_xor_block(block, ((uint8_t*)message)+i);
        aes128_enc(block, &ctx);
    }

    // szyfrowanie ostatniego bloku osobnym kluczem
    memset(&ctx, 0, sizeof(aes128_ctx_t));
    aes128_init(enc_key, &ctx);
    aes128_enc(block, &ctx);
}
\end{lstlisting}
\end{figure}

% itd.

\listoffigures
\printbibliography[heading=bibintoc]

\end{document}
