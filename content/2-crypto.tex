\chapter{Metody uwierzytelniania}
\label{cha:metodyUwierzytelniania}

W zależności od potrzeb i ograniczeń stosuje się różne metody uwierzytelniania podmiotów w komunikacji. Wyróżnić należy uwierzytelnianie przy pomocy kryptografii asymetrycznej, w której używana jest para matematycznie związanych ze sobą kluczy, oraz uwierzytelnianie przy pomocy kryptografii symetrycznej, w której używany jest jeden, współdzielony, tajny klucz.

Klucze w przypadku kryptografii asymetrycznej muszą posiadać konkretne właściwości. W przypadku algorytmu RSA bezpieczeństwo polega na trudności w faktoryzowaniu dużych liczb, co wymaga stosowania kluczy co najmniej 2048 bitowych~\cite{Nist}. Klucze w przypadku kryptografii symetrycznej nie muszą mieć konkretnych właściwości poza ich nieprzewidywalnością.

Ważną różnicą jest też wydajność. Kryptografia asymetryczna jest dużo bardziej złożona obliczeniowo od symetrycznej~\cite{al2008comparative}. Jest to szczególnie istotne na ograniczonych sprzętowo systemach wbudowanych. Przewagą kryptografii asymetrycznej jest jednak brak konieczności ustalenia wspólnego klucza przed rozpoczęciem komunikacji, jak ma to miejsce w przypadku kryptografii symetrycznej.

Zalecanym rozwiązaniem jest najpierw ustalenie wspólnego, tajnego klucza przy użyciu kryptografii asymetrycznej, a następnie użycie tego klucza do kryptografii symetrycznej~\cite{al2008comparative}.

\section{Kryptografia asymetryczna}
\label{sec:kryptoAsym}

Przy wyborze algorytmu dla potrzeb pracy istotne były:

\begin{itemize}
\item jakość implementacji dostępnych na mikroprocesory AVR,
\item złożoność obliczeniowa (niższa jest lepsza),
\item długość klucza wymagana do zapewnienia niezbędnego poziomu bezpieczeństwa.
\end{itemize}

Biblioteka \emph{AVR-Crypto-Lib} dostarcza implementację algorytmów RSA oraz DSA\footnote{\url{https://trac.cryptolib.org/avr-crypto-lib/browser}}. Biblioteka \emph{Emsign} dostarcza implementację RSA, lecz tylko z 64 bitowym kluczem\footnote{\url{http://www.emsign.nl/}}, co nie jest wystarczające dla zapewnienia bezpieczeństwa. Komercyjna biblioteka \emph{LightCrypt-AVR8-ECC} oraz biblioteka \emph{micro-ecc} dostarczają implementację kryptografii opartej o krzywe eliptyczne\footnote{\url{http://industrial.crypto.cmmsigma.eu/lightcrypt_avr8/lc_avr8_ecc.pl.html}}. Brak jest na rynku implementacji innych algorytmów klucza publicznego. Dostępność implementacji ogranicza wybór algorytmu do RSA, DSA oraz krzywych eliptycznych.

Następnym kryterium jest złożoność obliczeniowa. W analizie przeprowadzonej przez pracowników \emph{Sun Microsystems Laboratories} wykazano, że na mikroprocesorach AVR algorytmy oparte o krzywe eliptyczne są o rząd wielkości szybsze od algorytmu RSA~\cite{Gura2004}.

Krzywe eliptyczne wymagają najkrótszych kluczy. Rekomendacje \gls{nist}~\cite{Nist} podają, że 256-bitowy klucz \gls{ecc} zapewnia bezpieczeństwo porównywalne do 3072-bitowego klucza RSA lub DSA.

W związku z przewagą krzywych eliptycznych przy zadanych założeniach do ustalenia wspólnego klucza wybrano algorytm \gls{ecdh}. Wadą tego rozwiązania jest niezmienność klucza ustalanego tą metodą. Powoduje to brak utajnienia przekazywania \emph{(ang. forward secrecy)}.

\section{Kryptografia symetryczna}
\label{sec:kryptoSym}

ECBC-MAC
OMAC
CCM
HMAC
