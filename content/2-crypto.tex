\chapter{Metody uwierzytelniania}
\label{cha:metodyUwierzytelniania}

W zależności od potrzeb i ograniczeń stosuje się różne metody uwierzytelniania podmiotów w komunikacji. Wyróżnić należy uwierzytelnianie przy pomocy kryptografii asymetrycznej, w której używana jest para matematycznie związanych ze sobą kluczy, oraz uwierzytelnianie przy pomocy kryptografii symetrycznej, w której używany jest jeden, współdzielony, tajny klucz.

Klucze w przypadku kryptografii asymetrycznej muszą posiadać konkretne właściwości. W przypadku algorytmu RSA bezpieczeństwo polega na trudności w faktoryzowaniu dużych liczb, co wymaga stosowania kluczy co najmniej 2048 bitowych~\cite{Nist}. Klucze w przypadku kryptografii symetrycznej nie muszą mieć konkretnych właściwości poza ich nieprzewidywalnością.

Ważną różnicą jest też wydajność. Kryptografia asymetryczna jest dużo bardziej złożona obliczeniowo od symetrycznej~\cite{al2008comparative}. Jest to szczególnie istotne na ograniczonych sprzętowo systemach wbudowanych. Przewagą kryptografii asymetrycznej jest jednak brak konieczności ustalenia wspólnego klucza przed rozpoczęciem komunikacji, jak ma to miejsce w przypadku kryptografii symetrycznej.

Zalecanym rozwiązaniem jest najpierw ustalenie wspólnego, tajnego klucza przy użyciu kryptografii asymetrycznej, a następnie użycie tego klucza do kryptografii symetrycznej~\cite{al2008comparative}.

\section{Kryptografia asymetryczna}
\label{sec:kryptoAsym}

Przy wyborze algorytmu używanego do ustalania klucza dla potrzeb pracy istotne były:

\begin{itemize}
\item jakość implementacji algorytmów dostępnych na mikroprocesory AVR,
\item złożoność obliczeniowa,
\item długość klucza wymagana do zapewnienia niezbędnego poziomu bezpieczeństwa.
\end{itemize}

Biblioteka \emph{AVR-Crypto-Lib} dostarcza implementację algorytmów RSA oraz DSA\footnote{\url{https://trac.cryptolib.org/avr-crypto-lib/browser}}. Biblioteka \emph{Emsign} dostarcza implementację RSA, lecz tylko z 64 bitowym kluczem\footnote{\url{http://www.emsign.nl/}}, co nie jest wystarczające dla zapewnienia bezpieczeństwa. Komercyjna biblioteka \emph{LightCrypt-AVR8-ECC} oraz biblioteka \emph{micro-ecc} dostarczają implementację kryptografii opartej o krzywe eliptyczne\footnote{\url{http://industrial.crypto.cmmsigma.eu/lightcrypt_avr8/lc_avr8_ecc.pl.html}}. Brak jest na rynku implementacji innych algorytmów klucza publicznego. Dostępność implementacji ogranicza wybór algorytmu do RSA, DSA oraz krzywych eliptycznych.

Następnym kryterium jest złożoność obliczeniowa. W analizie przeprowadzonej przez pracowników \emph{Sun Microsystems Laboratories} wykazano, że na mikroprocesorach AVR algorytmy oparte o krzywe eliptyczne są o rząd wielkości szybsze od algorytmu RSA~\cite{Gura2004}.

Krzywe eliptyczne wymagają najkrótszych kluczy. Rekomendacje NIST~\cite{Nist} (National Institute of Standards and Technology) podają, że 256-bitowy klucz ECC \emph{(ang. Elliptic Curve Cryptography)} zapewnia bezpieczeństwo porównywalne do 3072-bitowego klucza RSA lub DSA.

W związku z przewagą krzywych eliptycznych przy zadanych założeniach do ustalenia wspólnego klucza wybrano algorytm ECDH \emph{(ang. Elliptic Curve Diffie-Hellman)}. Wadą tego rozwiązania jest niezmienność klucza ustalanego tą metodą. Powoduje to brak utajnienia przekazywania \emph{(ang. forward secrecy)}.

\section{Kryptografia symetryczna}
\label{sec:kryptoSym}

Przy wyborze algorytmu dla potrzeb pracy istotne były:

\begin{itemize}
\item jakość implementacji dostępnych na mikroprocesory AVR,
\item możliwość szyfrowania i uwierzytelniania danych.
\end{itemize}

Powszechnie dostępne są jedynie implementacje samych blokowych algorytmów szyfrowania takich jak AES oraz DES lub funkcji skrótu takich jak SHA-256. By uzyskać uwierzytelnianie wiadomości o zmiennej długości trzeba algorytmy blokowe zastosować w odpowiedni sposób. Przykładem jest tryb CBC-MAC {\itshape (ang. Cipher Block Chaining - Message Authentication Code)}. Pozwala on na wygenerowanie kodu uwierzytelniającego daną wiadomość, poprzez zaszyfrowanie jej w trybie CBC i użycie ostatniego bloku szyfrogramu jako kodu.

Tryb CBC-MAC -- przy nieprawidłowej implementacji -- może wprowadzić podatności:

\begin{itemize}
    \item użycie zmiennego wektora inicjalizacyjnego i przesyłanie go wraz z uwierzytelnianą wiadomością pozwala na dowolną modyfikację pierwszego bloku (16 bajtów) wiadomości bez zmiany kodu uwierzytelniającego,
    \item użycie tego samego klucza do szyfrowania w trybie CBC oraz uwierzytelniania w trybie CBC-MAC pozwala na obliczenie użytego klucza bez jego wcześniejszej znajomości,
    \item atakujący znający dwie wiadomości $ m $ oraz $ m' $ oraz ich kody uwierzytelniające może policzyć klucz uwierzytelniający wiadomości będącej specyficznym połączeniem wiadomości $ m $ oraz $ m' $.
\end{itemize}

Wszystkim tym podatnościom da się zapobiec poprzez użycie niezmiennego wektora inicjalizacyjnego oraz zaszyfrowanie ostatniego bloku innym kluczem (tryb ECBC-MAC, {\itshape ang. Encrypt-last-block CBC-MAC}).

Alternatywą jest także zastosowanie HMAC {\itshape (ang. keyed-hash message authentication code)}. Kodem uwierzytelniającym jest wtedy wynik funkcji skrótu policzony z połączenia współdzielonego klucza oraz uwierzytelnianej wiadomości \cite{krawczyk1997hmac}.

W pracy do uwierzytelniania wybrano AES w trybie ECBC-MAC. Zaletą tego rozwiązania jest możliwość użycia tej samej implementacji trybu CBC zarówno do szyfrowania jak i jako element trybu ECBC-MAC.

W implementacji szyfrowania w trybie CBC należało rozwiązać problemy wymienione poniżej.

\begin{enumerate}
    \item Użycie przewidywalnych wektorów inicjalizacyjnych pozwala atakującemu na zgadywanie treści wiadomości, a następnie -- poprzez odpowiednie spreparowanie nowej wiadomości -- weryfikację, czy wiadomość się zgadza. Wektory inicjalizacyjne muszą być nieprzewidywalne.
    \item CBC operuje na blokach danych, a więc wiadomości o długości niebędącej wielokrotnością długości bloku trzeba dopełniać. Oznacza to że do szyfrowanej wiadomości trzeba dołączać jej długość lub użyć dopełnienia, które jest jednoznaczne.
\end{enumerate}

Szyfrowanie z uwierzytelnianiem jest połączone wedle zasady {\itshape Encrypt-then-MAC}. Oznacza to że wiadomość najpierw jest szyfrowana, a następnie uwierzytelniany jest szyfrogram, a nie bezpośrednio wiadomość. Jest to rozwiązanie zapewniające najwyższe bezpieczeństwo, zapobiegające między innymi atakom typu \emph{padding oracle}~\cite{black2011authenticated}. Całość procesu została przedstawiona na rysunku \ref{fig:etm}.

\begin{figure}[h]
\centering
\includegraphics[width=0.7\textwidth]{images/etm.png}
\caption{Proces szyfrowania i uwierzytelniania danych}
\label{fig:etm}
\end{figure}
