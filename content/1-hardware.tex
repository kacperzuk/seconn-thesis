\chapter{Charakterystyka platformy sprzętowej}
\label{cha:hardware}

\begin{table}[h]
\centering
\caption{Wybrane modele AVR wraz z ich parametrami}
\begin{tabular}{|l|l|l|l|l|}
    \hline
    \textbf{Nazwa}  &
    \textbf{SRAM\footnote{ang. Static Random Access Memory}}  &
    \textbf{Wymagane napięcie}  &
    \textbf{Taktowanie procesora}  &
    \textbf{Liczba linii I/O} \\
    \hline
    ATtiny4 \cite{Attiny4}& 32 B & 1.8 - 5.5 V & do 12 MHz & 4\\
    \hline
    ATmega32u4 \cite{Atmega32} & 2,5 KB & 2.7 - 5.5 V & do 16 MHz & 26\\
    \hline
    ATxmega384C3 \cite{Atxmega384} & 32 KB & 1.6 - 3.6 V & do 32 MHz & 50\\
    \hline
\end{tabular}
\label{tab:avrmodels}
\end{table}

Mikropocesory Atmel AVR są w większości 8-bitowe i na takich skupia się praca. Rodzina AVR jest szeroka, kilka wybranych modeli przedstawiono w tabeli \ref{tab:avrmodels}. W pracy wykorzystany został model ATmega32u4 z 2,5 kilobajta SRAM~\cite{Atmega32}.

SRAM jest głównym ograniczeniem w implementacji uwierzytelniania, ponieważ 32 bajty nie są wystaczające do przeprowadzania operacji kryptograficznych, przy których sam klucz zajmuje 16 lub 32 bajty. Należy też pamiętać, że obsługa bezpiecznego połączenia nie może zajmować całości pamięci. Część pamięci należy przeznaczyć na obsługę peryferiów oraz właściwą logikę programu.

Istotnym elementem jest też wielkość domyślnych buforów. \emph{Arduino} w modułach \emph{Serial} oraz \emph{SoftwareSerial} domyślnie używa 16- lub 64-bajtowego (w zależności od ilości dostępnej pamięci) buforu na przychodzące dane\footnote{https://github.com/arduino/Arduino/blob/master/hardware/arduino/avr/cores/arduino/HardwareSerial.h}. Przy wiadomościach dłuższych niż 32 bajty oznacza to, że zbyt długie przetwarzanie jednej wiadomości spowoduje błędne odebranie następnej, jeżeli zostanie ona za szybko wysłana.

Maksymalne taktowanie mikroprocesora zależy od konkretnego modelu (od 12 do 32 MHz) oraz napięcia zasilania. Wykorzystywany w pracy Atmega32u4 zasilany był napięciem 5 V, co przekłada się na taktowanie 16 MHz.
