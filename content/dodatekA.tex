\chapter{Przykładowe rekordy protokołu komunikacji}
\label{app:samplerecords}

\begin{table}[h]
\centering
\caption{Budowa rekordu typu HelloRequest}
\begin{BVerbatim}
# Nagłówek:
0x00 0x01 # wersja protokołu
0x00      # typ rekordu: HelloRequest
0x00 0x40 # długość zawartości: 64 bajty

# Zawartość:
# 32 bajty współrzędnej X klucza publicznego
0x12 0x34 0x56 0x78 0x9A 0xBC 0xDE 0xF0
0x12 0x34 0x56 0x78 0x9A 0xBC 0xDE 0xF0
0x12 0x34 0x56 0x78 0x9A 0xBC 0xDE 0xF0
0x12 0x34 0x56 0x78 0x9A 0xBC 0xDE 0xF0

# 32 bajty współrzędnej Y klucza publicznego
0x12 0x34 0x56 0x78 0x9A 0xBC 0xDE 0xF0
0x12 0x34 0x56 0x78 0x9A 0xBC 0xDE 0xF0
0x12 0x34 0x56 0x78 0x9A 0xBC 0xDE 0xF0
0x12 0x34 0x56 0x78 0x9A 0xBC 0xDE 0xF0
\end{BVerbatim}
\label{fig:hellorequestsample}
\end{table}

\begin{table}
\centering
\caption{Budowa rekordu typu HelloResponse}
\begin{BVerbatim}
# Nagłówek:
0x00 0x01 # wersja protokołu
0x01      # typ rekordu: HelloResponse
0x00 0x60 # długość zawartości: 96 bajtów

# Zawartość:
# 16 bajtów kodu uwierzytelniającego
0x12 0x34 0x56 0x78 0x9A 0xBC 0xDE 0xF0
0x12 0x34 0x56 0x78 0x9A 0xBC 0xDE 0xF0

# Zaszyfrowany klucz publiczny (64 bajty)
# z uwzględnieniem dopełnienia PKCS#7 (16 bajtów)
0x12 0x34 0x56 0x78 0x9A 0xBC 0xDE 0xF0
0x12 0x34 0x56 0x78 0x9A 0xBC 0xDE 0xF0
0x12 0x34 0x56 0x78 0x9A 0xBC 0xDE 0xF0
0x12 0x34 0x56 0x78 0x9A 0xBC 0xDE 0xF0
0x12 0x34 0x56 0x78 0x9A 0xBC 0xDE 0xF0
0x12 0x34 0x56 0x78 0x9A 0xBC 0xDE 0xF0
0x12 0x34 0x56 0x78 0x9A 0xBC 0xDE 0xF0
0x12 0x34 0x56 0x78 0x9A 0xBC 0xDE 0xF0
0x12 0x34 0x56 0x78 0x9A 0xBC 0xDE 0xF0
0x12 0x34 0x56 0x78 0x9A 0xBC 0xDE 0xF0
\end{BVerbatim}
\label{fig:helloresponsesample}
\end{table}

\begin{table}
\centering
\caption{Budowa rekordu typu EncryptedData}
\begin{BVerbatim}
# Nagłówek:
0x00 0x01 # wersja protokołu
0x02      # typ rekordu: EncryptedData
0x00 0x20 # długość zawartości: 32 bajty

# Zawartość:
# 16 bajtów kodu uwierzytelniającego
0x12 0x34 0x56 0x78 0x9A 0xBC 0xDE 0xF0
0x12 0x34 0x56 0x78 0x9A 0xBC 0xDE 0xF0

# Zaszyfrowane dane (dopełnione do pełnego bloku AES)
0x12 0x34 0x56 0x78 0x9A 0xBC 0xDE 0xF0
0x12 0x34 0x56 0x78 0x9A 0xBC 0xDE 0xF0
0x12 0x34 0x56 0x78 0x9A 0xBC 0xDE 0xF0
0x12 0x34 0x56 0x78 0x9A 0xBC 0xDE 0xF0
\end{BVerbatim}
\label{fig:encrypteddatasample}
\end{table}
