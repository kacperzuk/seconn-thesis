\chapter{Test opracowanej biblioteki}
\label{cha:walidacja}

Zweryfikowane zostały trzy aspekty:

\begin{itemize}
\item poprawność zaprojektowania interfejsu programistycznego stworzonej biblioteki,
\item możliwość poprawnego nawiązania połączenia i przesłania danych między dwoma węzłami,
\item poprawność implementacji algorytmów kryptograficznych.
\end{itemize}

\section{Poprawność interfejsu programistycznego}

Poprawny interfejs programistyczny biblioteki musi umożliwiać implementację pełnego rozwiązania zapewniającego bezpieczną komunikację. Zostało to zweryfikowane poprzez~stworzenie przykładowego oprogramowania wykorzystującego bibliotekę. Oprogramowanie to powstało na~platformę Arduino oraz~wykorzystuje moduł Bluetooth XM-15B do~komunikacji.

Po~uruchomieniu urządzenia inicjalizowana jest stworzona biblioteka implementujące bezpieczną komunikację, a~moduł Bluetooth zostaje skonfigurowany w~trybie \emph{slave} z~nazwą ,,seconn'' i~oczekuje na~połączenie. Argumentami funkcji inicjalizującej bibliotekę są wskaźniki na~następujące funkcje zaimplementowane w~przykładowym oprogramowaniu:

\begin{itemize}
    \item funkcja obsługująca przekazywanie danych z~biblioteki do~modułu Bluetooth celem przesłania do~drugiego węzła,
    \item funkcja obsługująca i~przekazująca połączeniem szeregowym dane przychodzące z~biblioteki, które~zostały przez bibliotekę poprawnie uwierzytelnione oraz~zdeszyfrowane,
    \item funkcja obsługująca powiadomienia o~zmianie stanu połączenia przychodzące z~biblioteki i~przekazująca połączeniem szeregowym uwierzytelniony klucz publiczny drugiego węzła po~nawiązaniu bezpiecznego połączenia,
    \item funkcja generująca liczby losowe stworzona w~oparciu o~implementację zaproponowaną w~bibliotece {\itshape micro-ecc} (opis w~dodateku~\ref{app:randgen}).
\end{itemize}

Dodatkowo zaimplementowane zostało przekazywanie połączeniem szeregowym klucza publicznego urządzenia celem weryfikacji z~drugim węzłem oraz~przekazywanie danych przychodzących z~modułu Bluetooth do~biblioteki. Przepływ danych między przykładowym oprogramowaniem, biblioteką, drugim węzłem oraz~połączeniem szeregowym przedstawiono w~Dodatku~\ref{app:samplediagram}.

Należy zwrócić uwagę, że~oprogramowanie korzystające z~biblioteki nie implementuje żadnej logiki związanej z~protokołem komunikacji. Jest ona w~całości zaimplementowana w~bibliotece, a~oprogramowanie jedynie zajmuje się przekazywaniem danych między fizycznym połączeniem a~biblioteką. Całość oprogramowania wraz z~funkcją generującą liczby losowe składa się ze~103 linii kodu, ośmiu funkcji i~trzech plików.

W~tabeli \ref{tab:sample-comm} przedstawiono przykładowe dane, jakie zostają przesłane przez połączenie szeregowe. W~tym przypadku drugi węzeł był odpowiedzialny za~rozpoczęcie połączenia (przesłanie pierwszego HelloRequest), a~po poprawnym uwierzytelnieniu przesłał uwierzytelnioną i~zaszyfrowaną wiadomość o~treści ,,Some message...''.

\begin{table}
\centering
\caption{Przykładowe dane przesłane przez połączenie szeregowe. Stan nr 4 oznacza, że odebrany został prawidłowo uwierzytelniony pakiet HelloResponse zawierający klucz publiczny drugiego węzła.}
\begin{BVerbatim}
S!
Our pubkey is: 0x6D35D8BE2F0C67210C143E649F250FC4E
B014F25C305AC7C2FA6B02F0B4A4E63EA0BB52367AAF96E63B
BD968C186830ADE2B2A24769CB32E1E1A690F51079C7E
State:4
Pubkey of other side is: 0x22743237010F6830994886B
BFB781184C10D25E1D6819D075F40CF0724FEC049FF4804F82
58C14049E373595BC0987061B93493E16C8C59E8C7C2A64FF5
247B0
D:>Some message...<
\end{BVerbatim}
\label{tab:sample-comm}
\end{table}

\section{Poprawność implementacji algorytmów kryptograficznych}

W~trakcie tworzenia implementacje algorytmów były weryfikowane poprzez~porównywanie przykładowych zestawów danych wejściowych i~wejściowych z~niezależnymi implementacjami.

Algorytm AES oraz~tryb CBC zostały przetestowane przy użyciu aplikacji internetowej \emph{OnlineDomainTools}\footnote{http://aes.online-domain-tools.com/ [dostęp 5.01.2017]}. Ze~względu na~brak dokumentacji dotyczącej dopełniania do~pełnych bloków w~aplikacji \emph{OnlineDomainTools}, dane wejściowe podawano już dopełnione. Zweryfikowano:

\begin{itemize}
    \item zgodność tekstu jawnego po~zaszyfrowaniu na~urządzeniu AVR i~zdeszyfrowaniu w~aplikacji,
    \item zgodność tekstu jawnego po~zaszyfrowaniu w~aplikacji i~zdeszyfrowaniu na~urządzeniu AVR,
    \item zgodność szyfrogramów po~zaszyfrowaniu tekstu jawnego w~aplikacji oraz~na~urządzeniu AVR przy zastosowaniu tego samego wektora inicjalizacyjnego.
\end{itemize}

Wszystkie powyższe testy wykonano zarówno dla danych wejściowych o~długości jednego jak~i~wielu bloków.

Poprawność ustalenie wspólnego sekretu algorytmem ECDH sprawdzono poprzez~porównanie sekretu obliczonego na~urządzeniu z~sekretem obliczonym przez aplikację internetową \emph{JavaScript ECDH Key Exchange Demo}\footnote{http://www-cs-students.stanford.edu/~tjw/jsbn/ecdh.html [dostęp 5.01.2017]}. Operacje dokonywano na~kluczach wygenerowanych na~urządzeniu AVR przez bibliotekę \emph{micro-ecc}.

Poprawność implementacji algorytmów CBC i~dopełniania według PKCS\#7 stworzonych w~ramach pracy oraz~algorytmów AES, SHA-256 oraz~ECDH dostarczonych przez zewnętrzne biblioteki została także zweryfikowana poprzez~stworzenie drugiej implementacji protokołu w~języku Java.

Implementacje algorytmów AES, CBC, dopełniania według PKCS\#7, SHA-256 oraz~ECDH pochodzą z~pakietów java.security oraz~javax.crypto biblioteki standardowej języka Java. Implementacja ECBC-MAC została wykonana w~ramach pracy w~oparciu o~implementację CBC dostarczoną przez bibliotekę standardową. Użyte konfiguracje szyfrów to \emph{AES/CBC/NoPadding} i~\emph{AES/ECB/NoPadding} do~obliczania sygnatury oraz~\emph{AES/CBC/PKCS7Padding} do~szyfrowania i~deszyfrowania.

W~oparciu o~tak stworzoną bibliotekę w~języku Java napisano przykładową aplikację na~platformę Android. Aplikacja po~uruchomieniu stara się nawiązać połączenie Bluetooth z~urządzeniem o~nazwie ,,seconn''. Po~nawiązaniu połączenia Bluetooth wywoływana jest metoda biblioteki służąca nawiązaniu bezpiecznego połączenia (wysłaniu pierwszego HelloRequest). Wszystkie zmiany stanu połączenia są na~bieżąco wyświetlane na~ekranie, a~po nawiązaniu bezpiecznego połączenia wyświetlane są klucze publiczne obu węzłów i~możliwe jest przesyłanie uwierzytelnionych i~zaszyfrowanych danych do~drugiego węzła.

\begin{figure}
\centering
\includegraphics[height=0.6\textheight]{images/android.png}
\caption{Zrzut ekranu z przykładowej aplikacji implementującej protokół na platformę Android}
\label{fig:android}
\end{figure}

Przykładowa aplikacja mobilna skutecznie połączyła się z~urządzeniem AVR, a~przesyłane dane były prawidłowo deszyfrowane, co potwierdza że~implementacja algorytmów z~zewnętrzych bibliotek AVR oraz~implementacje wykonane w~ramach pracy są zgodne z~referencyjnymi implementacjami dostępnymi w~bibliotece standardowej języka Java na~platformie Android. Zrzut ekranu z~aplikacji przedstawiono na~rysunku \ref{fig:android}.

