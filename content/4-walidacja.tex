\chapter{Test opracowanej biblioteki}
\label{cha:walidacja}

Walidacji poddane zostały trzy aspekty:

\begin{itemize}
\item poprawność zaprojektowania interfejsu programistycznego stworzonej biblioteki
\item możliwość poprawnego nawiązania połączenia i przesłania danych między dwoma węzłami
\item poprawność implementacji algorytmów kryptograficznych
\end{itemize}

\section{Poprawność interfejsu programistycznego}

Poprawny interfejs programistyczny biblioteki musi umożliwiać stworzenie pełnego rozwiązania zapewniającego bezpieczną komunikację. Zostało to zweryfikowane poprzez stworzenie przykładowego oprogramowania wykorzystującego bibliotekę. Oprogramowanie to powstało na platformę Arduino oraz wykorzystuje moduł Bluetooth XM-15B.

Po uruchomieniu urządzenia biblioteka implementujące bezpieczną komunikację jest inicjalizowana, a moduł Bluetooth zostaje skonfigurowany w trybie \emph{slave} i oczekuje na połączenie. Przy inicjalizacji biblioteki przekazywane są następujące funkcje zaimplementowane w przykładowym oprogramowaniu:

\begin{itemize}
    \item funkcja obsługująca przekazywanie danych z biblioteki do modułu Bluetooth celem przesłania do drugiego węzła,
    \item funkcja obsługująca i przekazująca połączeniem szeregowym dane przychodzące z biblioteki, które zostały przez bibliotekę poprawnie uwierzytelnione oraz zdeszyfrowane,
    \item funkcja obsługująca powiadomienia o zmianie stanu połączenia przychodzące z biblioteki i przekazująca połączeniem szeregowym uwierzytelniony klucz publiczny drugiego węzła po nawiązaniu bezpiecznego połączenia,
    \item funkcja generująca liczby losowe stworzona w oparciu o implementację zaproponowaną w bibliotece {\itshape micro-ecc} (Dodatek~\ref{app:randgen}).
\end{itemize}

Dodatkowo zaimplementowane zostało przekazywanie połączeniem szeregowym klucza publicznego urządzenia celem weryfikacji z drugim węzłem oraz przekazywanie danych przychodzących z modułu Bluetooth do biblioteki. Przepływ danych między przykładowym oprogramowaniem, biblioteką, drugim węzłęm oraz połączeniem szeregowym przedstawiono w Dodatku~\ref{app:samplediagram}.

Należy zwrócić uwagę, że oprogramowanie korzystające z biblioteki nie implementuje żadnej logiki związanej z protokołem komunikacji. Jest ona w całości zaimplementowana w bibliotece, a oprogramowanie jedynie zajmuje się przekazywaniem danych między fizycznym połączeniem a biblioteką.

W tabeli \ref{tab:sample-comm} przedstawiono przykładowe dane, jakie zostają przesłane przez połączenie szeregowe. W tym przypadku drugi węzeł był odpowiedzialny za rozpoczęcie połączenia (przesłanie pierwszego HelloRequest), a po poprawnym uwierzytelnieniu przesłał uwierzytelnioną i zaszyfrowaną wiadomość o treści ,,Some message...''.

\begin{table}
\centering
\caption{Przykładowe dane przesłane przez połączenie szeregowe. Stan nr 4 oznacza, że odebrany został prawidłowo uwierzytelniony pakiet HelloResponse zawierający klucz publiczny drugiego węzła.}
\begin{BVerbatim}
S!
Our pubkey is: 0x6D35D8BE2F0C67210C143E649F250FC4E
B014F25C305AC7C2FA6B02F0B4A4E63EA0BB52367AAF96E63B
BD968C186830ADE2B2A24769CB32E1E1A690F51079C7E
State:4
Pubkey of other side is: 0x22743237010F6830994886B
BFB781184C10D25E1D6819D075F40CF0724FEC049FF4804F82
58C14049E373595BC0987061B93493E16C8C59E8C7C2A64FF5
247B0
D:>Some message...<
\end{BVerbatim}
\label{tab:sample-comm}
\end{table}

\section{Poprawność implementacji algorytmów kryptograficznych}

Poprawność implementacji algorytmów \gls{cbc} i dopełniania według PKCS\#7 stworzonych w ramach pracy oraz algorytmów AES, SHA-256 oraz ECDH dostarczonych przez zewnętrzne biblioteki została zwalidowana poprzez stworzenie drugiej implementacji protokołu w języku Java.

Implementacje algorytmów AES, CBC, dopełniania według PKCS\#7, SHA-256 oraz ECDH pochodzą z pakietu java.security biblioteki standardowej języka Java. Implementacja ECBC-MAC została wykonana w ramach pracy w oparciu o implementację CBC dostarczoną przez bibliotekę standardową.
