\chapter{Przykładowe bloki kodu biblioteki}
\label{app:codesamples}

\definecolor{mygreen}{rgb}{0,0.6,0}
\definecolor{mygray}{rgb}{0.5,0.5,0.5}
\definecolor{mymauve}{rgb}{0.58,0,0.82}

\lstset{%
  basicstyle=\footnotesize\ttfamily,%
  keywordstyle=\bfseries\color{green!40!black},%
  commentstyle=\itshape\color{purple!40!black},%
  identifierstyle=\color{blue},%
  stringstyle=\color{orange},%
  breakatwhitespace=false,         % sets if automatic breaks should only happen at whitespace
  breaklines=true,                 % sets automatic line breaking
  captionpos=t,                    % sets the caption-position to bottom
  frame=single,	                   % adds a~frame around the code
  keepspaces=true,                 % keeps spaces in text, useful for keeping indentation of code (possibly needs columns=flexible)
  language=C,                 % the language of the code
  rulecolor=\color{black}         % if not set, the frame-color may be changed on line-breaks within not-black text (e.g. comments (green here))
}


\begin{table}[ht]
\caption{Szyfrowanie CBC wraz z obsługą dopełnienia PKCS\#7}
\label{lst:encrypt}
\begin{lstlisting}
size_t _seconn_crypto_encrypt(uint8_t *dest, uint8_t *src, size_t length, aes128_key_t enc_key) {
    rng(dest, 16); // random initialization vector
    memset(&ctx, 0, sizeof(aes128_ctx_t));

    aes128_init(enc_key, &ctx);

    aes128_enc(dest, &ctx);

    size_t i~= 0;
    for(; i+16 <= length; i~+= 16) {
        memcpy(dest+16+i, src+i, 16);
        _seconn_crypto_xor_block(dest+16+i, dest+i);
        aes128_enc(dest+16+i, &ctx);
    }

    size_t pad_length = 16 - (length % 16);
    memset(dest+16+i, pad_length, 16);
    memcpy(dest+16+i, src+i, length - i);
    _seconn_crypto_xor_block(dest+16+i, dest+i);
    aes128_enc(dest+16+i, &ctx);

    return i+32;
}
\end{lstlisting}
\end{table}

\begin{table}
\caption{Odszyfrowanie CBC wraz z obsługą dopełnienia PKCS\#7}
\label{lst:decrypt}
\begin{lstlisting}
size_t _seconn_crypto_decrypt(uint8_t *dest, uint8_t *src, size_t length, aes128_key_t enc_key) {
    memset(&ctx, 0, sizeof(aes128_ctx_t));
    aes128_init(enc_key, &ctx);

    size_t i~= 0;
    for(; i+16 < length; i~+= 16) {
        memcpy(dest+i, src+i+16, 16);
        aes128_dec(dest+i, &ctx);
        _seconn_crypto_xor_block(dest+i, src+i);
    }

    size_t pad_length = dest[i-1];
    for(size_t j = 2; j <= pad_length; j++) {
        if (dest[i-j] != pad_length) {
            return 0;
        }
    }

    return i-pad_length;
}
\end{lstlisting}
\end{table}

\begin{table}
\caption{Obliczanie MAC dla wiadomości}
\label{lst:mac}
\begin{lstlisting}
void _seconn_crypto_calculate_mac(uint8_t *mac, uint8_t *message,
size_t length, aes128_key_t mac_key, aes128_key_t enc_key) {
    memset(&ctx, 0, sizeof(aes128_ctx_t));
    aes128_init(mac_key, &ctx);

    uint8_t *block = mac;

    memset(block, 0, 16); // initialization vector filled with zeros

    // calculating last block of CBC encryption
    size_t i~= 0;
    for(; i+16 <= length; i~+= 16) {
        _seconn_crypto_xor_block(block, message+i);
        aes128_enc(block, &ctx);
    }

    // encrypting last block with separate key
    memset(&ctx, 0, sizeof(aes128_ctx_t));
    aes128_init(enc_key, &ctx);
    aes128_enc(block, &ctx);
}
\end{lstlisting}
\end{table}
