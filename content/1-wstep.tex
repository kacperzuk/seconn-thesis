\chapter{Wstęp}
\label{cha:wstep}

\begin{figure}[h]
\centering
\includegraphics[width=0.7\textwidth]{images/arduino-trends.png}
\caption{Relatywna liczba wyszukiwań frazy ,,Arduino'' w ostatnich pięciu latach. Źródło: Google Trends}
\label{fig:arduinotrends}
\end{figure}

AVR to rodzina mikroprocesorów opracowana i rozwijana przez firmę Atmel. Oparta o nią jest m. in. platforma Arduino, która -- jak przedstawiono na Rys. \ref{fig:arduinotrends} -- z roku na rok zyskuje popularność. Platforma Arduino zaprojektowana została z myślą o osobach, które niekoniecznie posiadają formalne wykształcenie inżynierskie~\cite{BanShi14}. Jest ona też często używana do prototypowania urządzeń, wpisujących się w koncepcję \emph{Internetu Rzeczy (ang. Internet of Things, IoT)}.

Urządzenia wbudowane podłączone do Internetu są szczególnie narażone na ataki. W 2016 roku podatne urządzenia wbudowane zostały wykorzystane do przeprowadzenia masywnych ataków typu DDoS~\cite{AkaIOT}.

\section{Cele pracy}

Istotne jest więc dostarczenie narzędzi, które pozwalają nie tylko na szybkie prototypowanie, ale które pozwolą także zachować odpowiedni poziom bezpieczeństwa. Należy pamiętać przede wszystkim o tym, że urządzenia \emph{IoT} są tworzone także przez ludzi bez formalnego wykształcenia inżynierskiego.

W niniejszej pracy przedstawiono protokół bezpiecznej komunikacji oraz bibliotekę programistyczną na urządzenia AVR zaprojektowane z myślą o prostocie obsługi. Wybrane zostały zestawy algorytmów, które zapewniają niezbędny poziom bezpieczeństwa. Ich złożoność została ukryta za prostym interfejsem programistycznym \emph{(ang. API)}, który nie pozwala na wprowadzenie błędów zmniejszających bezpieczeństwo.  Zaproponowane rozwiązanie zapewnia poufność, autentyczność oraz integralność przesyłanych danych.

\section{Zawartość pracy}

W rozdziale \cite{cha:hardware} scharakteryzowana jest platforma sprzętowa AVR, ze szczególnym uwzględnieniem jej ograniczeń. Następnie w rozdziale \cite{cha:metodyUwierzytelniania}
