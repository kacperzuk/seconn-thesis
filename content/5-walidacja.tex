\chapter{Walidacja}
\label{cha:walidacja}

Walidacji poddane zostały trzy aspekty:

\begin{itemize}
\item poprawność zaprojektowania interfejsu programistycznego stworzonej biblioteki
\item możliwość poprawnego nawiązania połączenia i przesłania danych między dwoma węzłami
\item poprawność implementacji algorytmów kryptograficznych
\end{itemize}

\section{Poprawność interfejsu programistycznego}

Poprawny interfejs programistyczny biblioteki musi umożliwiać stworzenie pełnego rozwiązania zapewniającego bezpieczną komunikację. Zostało to zweryfikowane poprzez stworzenie przykładowego oprogramowania wykorzystującego bibliotekę. Oprogramowanie to powstało na platformę Arduino oraz wykorzystuje moduł bluetooth XM-15B.

Po uruchomieniu urządzenia moduł bluetooth zostanie skonfigurowany w trybie \emph{slave} i 

\section{Poprawność komunikacji}

\section{Poprawność implementacji algorytmów}

Poprawność implementacji algorytmów \gls{cbc} i dopełniania według PKCS\#7 stworzonych w ramach pracy oraz algorytmów AES, SHA-256 oraz ECDH dostarczonych przez zewnętrzne biblioteki została zwalidowana poprzez stworzenie drugiej implementacji protokołu w języku Java.

Implementacje algorytmów AES, CBC, dopełniania według PKCS\#7, SHA-256 oraz ECDH pochodzą z pakietu java.security biblioteki standardowej języka Java. Implementacja ECBC-MAC została wykonana w ramach pracy w oparciu o implementację CBC dostarczoną przez bibliotekę standardową.
