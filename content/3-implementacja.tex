\chapter{Implementacja protokołu komunikacji}
\label{cha:implementacja}

Protokół komunikacji między dwoma węzłami zaprojektowano i zaimplementowano z następującymi założeniami:

\begin{itemize}
\item pełna funkcjonalność przy jak najmniejszych wymaganiach sprzętowych, w szczególności przy dostępnej małej ilości pamięci operacyjnej,
\item częściowa niezależność od warstwy sieciowej,
\item zapewnienie uwierzytelniania i szyfrowania wiadomości.
\end{itemize}

\section{Podstawowe struktury protokołu}
\label{sec:proto}

\begin{figure}[h]
\centering
\includegraphics[width=\textwidth]{images/wiadomosc.png}
\caption{Budowa wiadomości w protokole komunikacji.}
\label{fig:message-def}
\end{figure}

\begin{table}[t]
\centering
\caption{Typy wiadomości wraz z ich charakterystyką}
\begin{tabular}{|p{2.3cm}|p{1.4cm}|l|p{2.9cm}|p{3.1cm}|}
    \hline
    \textbf{Typ \mbox{wiadomości}}  &
    \textbf{Wartość pola typ}  &
    \textbf{Długość bloku danych}  &
    \textbf{Blok danych jest zaszyfrowany}  &
    \textbf{Blok danych jest uwierzytelniony}\\
    \hline
    HelloRequest & 0x00 & 64 bajty & nie & nie\\
    \hline
    HelloResponse & 0x01 & 96 bajtów & tak & tak\\
    \hline
    EncryptedData & 0x02 & zmienna, minimum 32 bajty & tak & tak\\
    \hline
\end{tabular}
\label{tab:recordtypes}
\end{table}

Podstawową jednostką protokołu są wiadomości zbudowane według schematu zaprezentowanego na Rys. \ref{fig:message-def}. Zdefiniowane typy wiadomości są przedstawione w tabeli \ref{tab:recordtypes}. Budowę przykładowych wiadomości przedstawiono na rysunkach \ref{fig:hellorequestsample}, \ref{fig:helloresponsesample} oraz \ref{fig:encrypteddatasample} w dodatku \ref{app:samplerecords}.

Odbiorca powinien zweryfikować:

\begin{itemize}
\item zgodność wersji protokołu -- wymagane bajty 0x00 oraz 0x01,
\item prawidłowość bajtu określającego typ -- wymagana wartość 0x00, 0x01 lub 0x02,
\item zgodność zadeklarowanej długości bloku danych z typem,
\item w przypadku typów HelloResponse oraz EncryptedData -- prawidłowość kodu uwierzytelniającego.
\end{itemize}

W przypadku niezgodności któregokolwiek elementu wiadomość powinna zostać zignorowana.

\section{Nawiązywanie połączenia}

\begin{figure}[h]
\centering
\begin{BVerbatim}
Nawiązujący połączenie              Odbierający połączenie
        |             HelloRequest              |
        |      zawiera KPN (klucz publiczny     |
        |       nawiązującego połączenie)       |
   1.   | +---------------------------------->> |
        |                                       |
        |             HelloRequest              |
        |      zawiera KPO (klucz publiczny     |
        |       odbierającego połączenie)       |
   2.   | <<----------------------------------+ |
        |                                       |
        |             HelloResponse             |
        |      zawiera uwierzytelniony KPN      |
   3.   | +---------------------------------->> |
        |                                       |
        |             HelloResponse             |
        |      zawiera uwierzytelniony KPO      |
   4.   | <<----------------------------------+ |
\end{BVerbatim}
\caption{Kolejność wymiany wiadomości w procesie nawiązywania połączenia}
\label{fig:handshake}
\end{figure}

Kolejność przesyłania wiadomości w celu nawiązania połączenia przedstawiona została na rysunku~\ref{fig:handshake}. HelloRequest zawiera klucz publiczny węzła, który go wysyła. Węzeł, który odbiera HelloRequest, używa swojego klucza publicznego oraz klucza publicznego z odebranej wiadomości do ustalenia sekretnego klucza. Nawiązującym połączenie może być dowolny węzeł.

Po ustaleniu wspólnego klucza węzły mogą wysłać HelloResponse, który zawiera zaszyfrowany i uwierzytelniony klucz publiczny pochodzący z wysyłającego węzła. Jeżeli węzeł odbierający wiadomość skutecznie potwierdzi, że jest ona prawidłowo uwierzytelniona, a zdeszyfrowany klucz publiczny pokrywa się z kluczem przesłanym wcześniej w HelloRequest, połączenie uznawane jest za nawiązane. Jeżeli przed odebraniem HelloResponse odebrany był więcej niż jeden HelloRequest, brana pod uwagę jest wiadomość odebrana jako ostatnia. Po nawiązaniu połączenia wymieniane mogą być tylko wiadomości typu EncryptedData.

Istotne jest, że protokół nie zapewnia autentyczności danego klucza publicznego. Powinno to zostać zweryfikowane niezależnie, na przykład poprzez wyświetlenie skrótu klucza użytkownikowi i poproszenie go o potwierdzenie, że na obu urządzeniach uczestniczących w komunikacji jest wyświetlony taki sam klucz.

Protokół zakłada też, że przesyłanie danych jest niezawodne, połączeniowe oraz zachowana jest ich kolejność. Nie są więc zaimplementowane retransmisje ani wykrywanie, czy drugi węzeł rzeczywiście nasłuchuje na przychodzące dane.

\section{Generowanie współdzielonego klucza}
\label{sec:sharedkey}

Każdy z węzłów po odebraniu HelloRequest używa odebranego klucza publicznego oraz swojego klucza publicznego do ustalenia wspólnego sekretu przy użyciu algorytmu ECDH \emph{(ang. Elliptic curve Diffie--Hellman)} oraz proponowanej przez NIST krzywej eliptycznej P-256~\cite{kerry2013digital} (w RFC 5480 nazwaną krzywą secp256r1~\cite{turner2009elliptic}).

Z sekretu będącego wynikiem algorytmu ECDH liczony jest skrót przy użyciu algorytmu SHA-256. Następnie jest on dzielony na dwie części po 128-bitów. Pierwsza część staje się współdzielonym kluczem używanym do szyfrowania, druga część staje się współdzielonym kluczem używanym do uwierzytelniania.

Implementacja algorytmu ECDH z krzywą eliptyczną P-256 pochodzi z biblioteki \emph{micro-ecc}. Jest to jedyna darmowa biblioteka implementująca ECDH.

\section{Szyfrowanie i deszyfrowanie danych}
\label{sec:encrypt}

Szyfrowanie bloku danych w wiadomości odbywa się za pomocą szyfru blokowego AES ze 128-bitowym kluczem używanym w trybie CBC. Wektor inicjalizacyjny jest losowy i dołączany do danych przesyłanej wiadomości przed szyfrogramem. Tekst jawny jest dopełniany do pełnego bloku według algorytmu zdefiniowanego w PKCS\#7~\cite{kaliski1998pkcs}.

Stworzona biblioteka nie posiada własnego źródła liczb losowych, musi zostać ono dostarczone w ramach integracji. Przykładowa metoda generowania liczb losowych na platformie Arduino została opisana w Dodatku \ref{app:randgen}.

Właściwe kroki potrzebne do zaszyfrowania danych wypisano poniżej.

\begin{enumerate}
\item Dopełnienie tekstu jawnego do pełnego bloku:
\begin{itemize}
\item jeżeli długość tekstu jawnego jest wielokrotnością długości bloku, do tekstu jawnego doklejone musi być 16 bajtów o wartości 16,
\item w przeciwnym wypadku, gdy wymagane jest dopełnienie $ N $ bajtów, do tekstu jawnego doklejone musi być $ N $ bajtów o wartości $ N $.
\end{itemize}
\item Zaszyfrowanie dopełnionego tekstu jawnego w trybie CBC z losowym wektorem inicjalizacyjnym.
\item Doklejenie wektora inicjalizacyjnego przed szyfrogramem.
\end{enumerate}

Przykłady dopełniania danych o różnych długościach przedstawiono w tabeli~\ref{tab:padding}.

\begin{table}[th]
\centering
\caption{Dopełnanie danych do pełnego bloku. Dopełnienie zaznaczone zostało kolorem niebieskiem i pogrubieniem.}
{\footnotesize Dane ,,Witaj swiecie'' (13 bajtów) zostają dopełnione 3 bajtami o wartości 3 (0x03):}

\texttt{0x57 0x69 0x74 0x61\\
0x6a 0x20 0x73 0x77\\
0x69 0x65 0x63 0x69\\
0x65 {\color[rgb]{0,0,1}\bfseries 0x03 0x03 0x03}}

{\footnotesize Dane ,,Witaj swiecie !!'' (16 bajtów) zostają dopełnione 16 bajtami o wartości 16 (0x10):}

\texttt{0x57 0x69 0x74 0x61\\
0x6a 0x20 0x73 0x77\\
0x69 0x65 0x63 0x69\\
0x65 0x20 0x21 0x21\\
{\color[rgb]{0,0,1}\bfseries
0x10 0x10 0x10 0x10\\
0x10 0x10 0x10 0x10\\
0x10 0x10 0x10 0x10\\
0x10 0x10 0x10 0x10\\
}
}

\label{tab:padding}
\end{table}

Kod implementujący szyfrowanie danych przedstawiono w tabeli~\ref{lst:encrypt} w dodatku~\ref{app:codesamples}.

Właściwe kroki potrzebne do odszyfrowania danych wypisano poniżej.

\begin{enumerate}
\item Oddzielenie wektora inicjalizacyjnego od szyfrogramu.
\item Zdeszyfrowanie szyfrogramu w trybie CBC przy wykorzystaniu oddzielonego wektora inicjalizacyjnego.
\item Pobranie wartości ostatniego bajtu zdeszyfrowanego ciągu:
\begin{itemize}
    \item wartość ta nazywana jest dalej $ N $.
\end{itemize}
\item Zweryfikowanie poprawności dopełnienia:
\begin{itemize}
    \item ostatnie $ N $ bajtów musi mieć wartość $ N $,
    \item jeżeli dopełnienie jest nieprawidłowe, cała wiadomość jest ignorowana.
\end{itemize}
\item Usunięcie ostatnich $ N $ bajtów.
\end{enumerate}

Kod implementujący odszyfrowanie danych przedstawiono w tabeli~\ref{lst:decrypt} w dodatku~\ref{app:codesamples}.

Implementacja algorytmu AES pochodzi z biblioteki \emph{AVR-Crypto-Lib}. Jest to najlepiej udokumentowana, darmowa biblioteka implementująca algorytm AES. Implementacja trybu CBC oraz algorytmu dopełniania zostały zrealizowane w ramach pracy.

\section{Uwierzytelnienie wiadomości}
\label{sec:auth}

%todo przyklad uwierzytelnienia

Uwierzytelnienie wiadomości odbywa się poprzez dołączenie do MAC do bloku danych. Dla danej wiadomości MAC tworzony jest za pomocą szyfru blokowego AES ze 128-bitowym kluczem używanym w trybie ECBC-MAC. Wektor inicjalizacyjny wypełniony jest zerami i nie jest przesyłany. Uwierzytelniany jest kompletny szyfrogram wraz z wektorem inicjalizacyjnym użytym do szyfrowania, a nie tekst jawny. Długość szyfrogramu wraz z wektorem inicjalizacyjnym zawsze będzie wielokrotnością długości bloku, a więc nie jest stosowane dopełnianie.

Tryb ECBC-MAC to tryb CBC-MAC, którego wynik jest dodatkowo szyfrowany innym kluczem niż ten użyty do CBC-MAC. W tej pracy do CBC-MAC użyty jest klucz służący do uwierzytelniania, a wynik CBC-MAC jest szyfrowany używając klucza służącego do szyfrowania.

Właściwe kroki potrzebne do obliczenia kodu uwierzytelniającego opisano poniżej.

\begin{enumerate}
\item Obliczenie ostatniego bloku będącego wynikiem zaszyfrowania szyfrogramu wraz z wektorem inicjalizacyjnym w trybie CBC z wektorem inicjalizacyjnym wypełnionym zerami przy użyciu klucza przeznaczonego do uwierzytelniania.
\item Zaszyfrowanie bloku przy wykorzystaniu AES i klucza przeznaczonego do szyfrowania.
\end{enumerate}

Węzeł wysyłający dokleja kod uwierzytelniający przed szyfrogramem. Węzeł odbierający oddziela otrzymany kod od szyfrogramu, oblicza kod uwierzytelniający dla danego szyfrogramu i porównuje, czy zgadza się on z kodem otrzymanym. Jeżeli kod obliczony różni się od kodu otrzymanego, cała wiadomość jest ignorowana.

Kod implementujący obliczanie MAC przedstawiono w tabeli~\ref{lst:mac} w dodatku~\ref{app:codesamples}.

Implementacja trybu ECBC-MAC została zrealizowana w ramach pracy.
