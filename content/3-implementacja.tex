\chapter{Implementacja protokołu komunikacji}
\label{cha:implementacja}

Protokół komunikacji między dwoma węzłami zaprojektowano i~zaimplementowano z~następującymi założeniami:

\begin{itemize}
\item pełna funkcjonalność przy jak najmniejszych wymaganiach sprzętowych, w szczególności przy dostępnej małej ilości pamięci operacyjnej,
\item częściowa niezależność od warstwy sieciowej,
\item zapewnienie uwierzytelniania i szyfrowania wiadomości.
\end{itemize}

\section{Charakterystyka platformy sprzętowej}
\label{cha:hardware}

Mikropocesory Atmel AVR są w~większości 8-bitowe i~na takich skupia się praca. Rodzina AVR jest szeroka, kilka wybranych modeli przedstawiono w~tabeli \ref{tab:avrmodels}. W~pracy wykorzystany został model ATmega32u4 z~2,5 kilobajta SRAM~\cite{Atmega32}.

\begin{table}[ht]
\centering
\caption{Wybrane modele AVR wraz z ich parametrami}
\begin{tabular}{|l|l|l|l|l|}
    \hline
    \textbf{Nazwa}  &
    \textbf{SRAM\footnote{ang. Static Random Access Memory}}  &
    \textbf{Wymagane napięcie}  &
    \textbf{Taktowanie procesora}  &
    \textbf{Liczba linii I/O} \\
    \hline
    ATtiny4 \cite{Attiny4}& 32 B & 1.8 - 5.5 V & do~12 MHz & 4\\
    \hline
    ATmega32u4 \cite{Atmega32} & 2,5 KB & 2.7 - 5.5 V & do~16 MHz & 26\\
    \hline
    ATxmega384C3 \cite{Atxmega384} & 32 KB & 1.6 - 3.6 V & do~32 MHz & 50\\
    \hline
\end{tabular}
\label{tab:avrmodels}
\end{table}

SRAM jest głównym ograniczeniem w~implementacji uwierzytelniania, ponieważ~32 bajty nie są wystaczające do~przeprowadzania operacji kryptograficznych, przy których sam klucz zajmuje 16 lub~32 bajty. Należy też pamiętać, że~obsługa bezpiecznego połączenia nie może zajmować całości pamięci. Część pamięci należy przeznaczyć na~obsługę peryferiów oraz~właściwą logikę programu.

Istotnym elementem jest też wielkość domyślnych buforów. \emph{Arduino} w~modułach \emph{Serial} oraz~\emph{SoftwareSerial} domyślnie używa 16- lub~64-bajtowego (w zależności od~ilości dostępnej pamięci) buforu na~przychodzące dane\footnote{https://github.com/arduino/Arduino/blob/master/hardware/arduino/avr/cores/arduino/HardwareSerial.h}. Przy wiadomościach dłuższych niż 32 bajty oznacza to, że~zbyt długie przetwarzanie jednej wiadomości spowoduje błędne odebranie następnej, jeżeli zostanie ona za~szybko wysłana.

\section{Podstawowe struktury protokołu}
\label{sec:proto}

\begin{figure}[ht]
\centering
\includegraphics[width=\textwidth]{images/wiadomosc.png}
\caption{Budowa wiadomości w protokole komunikacji.}
\label{fig:message-def}
\end{figure}

\begin{table}[t]
\centering
\caption{Typy wiadomości wraz z ich charakterystyką}
\begin{tabular}{|p{2.3cm}|p{1.4cm}|l|p{2.9cm}|p{3.1cm}|}
    \hline
    \textbf{Typ \mbox{wiadomości}}  &
    \textbf{Wartość pola typ}  &
    \textbf{Długość bloku danych}  &
    \textbf{Blok danych jest zaszyfrowany}  &
    \textbf{Blok danych jest uwierzytelniony}\\
    \hline
    HelloRequest & 0x00 & 64 bajty & nie & nie\\
    \hline
    HelloResponse & 0x01 & 96 bajtów & tak & tak\\
    \hline
    EncryptedData & 0x02 & zmienna, minimum 32 bajty & tak & tak\\
    \hline
\end{tabular}
\label{tab:recordtypes}
\end{table}

Podstawową jednostką protokołu są wiadomości zbudowane według schematu zaprezentowanego na~Rys. \ref{fig:message-def}. Zdefiniowane typy wiadomości są przedstawione w~tabeli \ref{tab:recordtypes}. Budowę przykładowych wiadomości przedstawiono na~rysunkach \ref{fig:hellorequestsample}, \ref{fig:helloresponsesample} oraz~\ref{fig:encrypteddatasample} w~dodatku \ref{app:samplerecords}.

Odbiorca powinien zweryfikować:

\begin{itemize}
\item zgodność wersji protokołu -- wymagane bajty 0x00 oraz 0x01,
\item prawidłowość bajtu określającego typ -- wymagana wartość 0x00, 0x01 lub 0x02,
\item zgodność zadeklarowanej długości bloku danych z typem,
\item w przypadku typów HelloResponse oraz EncryptedData -- prawidłowość kodu uwierzytelniającego.
\end{itemize}

W~przypadku niezgodności któregokolwiek elementu wiadomość powinna zostać zignorowana.

Narzut pamięci operacyjnej implementacji protokołu wynosi około 2 KB. Narzut na~rozmiar programu to 14.6 KB.

\section{Nawiązywanie połączenia}

\begin{figure}[ht]
\centering
\begin{BVerbatim}
Nawiązujący połączenie              Odbierający połączenie
        |             HelloRequest              |
        |      zawiera KPN (klucz publiczny     |
        |       nawiązującego połączenie)       |
   1.   | +---------------------------------->> |
        |                                       |
        |             HelloRequest              |
        |      zawiera KPO (klucz publiczny     |
        |       odbierającego połączenie)       |
   2.   | <<----------------------------------+ |
        |                                       |
        |             HelloResponse             |
        |      zawiera uwierzytelniony KPN      |
   3.   | +---------------------------------->> |
        |                                       |
        |             HelloResponse             |
        |      zawiera uwierzytelniony KPO      |
   4.   | <<----------------------------------+ |
\end{BVerbatim}
\caption{Kolejność wymiany wiadomości w procesie nawiązywania połączenia}
\label{fig:handshake}
\end{figure}

Kolejność przesyłania wiadomości w~celu nawiązania połączenia przedstawiona została na~rysunku~\ref{fig:handshake}. HelloRequest zawiera klucz publiczny węzła, który~go wysyła. Węzeł, który~odbiera HelloRequest, używa swojego klucza publicznego oraz~klucza publicznego z~odebranej wiadomości do~ustalenia sekretnego klucza. Nawiązującym połączenie może być dowolny węzeł.

Po~ustaleniu wspólnego klucza węzły mogą wysłać HelloResponse, który~zawiera zaszyfrowany i~uwierzytelniony klucz publiczny pochodzący z~wysyłającego węzła. Jeżeli węzeł odbierający wiadomość skutecznie potwierdzi, że~jest ona prawidłowo uwierzytelniona, a~zdeszyfrowany klucz publiczny pokrywa się z~kluczem przesłanym wcześniej w~HelloRequest, połączenie uznawane jest za~nawiązane. Jeżeli przed odebraniem HelloResponse odebrany był więcej niż jeden HelloRequest, brana pod~uwagę jest wiadomość odebrana jako ostatnia. Po~nawiązaniu połączenia wymieniane mogą być tylko wiadomości typu EncryptedData.

Istotne jest, że~protokół nie zapewnia autentyczności danego klucza publicznego. Powinno to zostać zweryfikowane niezależnie, na~przykład poprzez~wyświetlenie skrótu klucza użytkownikowi i~poproszenie go o~potwierdzenie, że~na obu urządzeniach uczestniczących w~komunikacji jest wyświetlony taki sam klucz.

Protokół zakłada też, że~przesyłanie danych jest niezawodne, połączeniowe oraz~zachowana jest ich kolejność. Nie są więc zaimplementowane retransmisje ani wykrywanie, czy~drugi węzeł rzeczywiście nasłuchuje na~przychodzące dane.

\section{Generowanie współdzielonego klucza}
\label{sec:sharedkey}

Każdy z~węzłów po~odebraniu HelloRequest używa odebranego klucza publicznego oraz~swojego klucza publicznego do~ustalenia wspólnego sekretu przy użyciu algorytmu ECDH \emph{(ang. Elliptic curve Diffie--Hellman)} oraz~proponowanej przez NIST krzywej eliptycznej P-256~\cite{kerry2013digital} (w RFC 5480 nazwaną krzywą secp256r1~\cite{turner2009elliptic}).

Z~sekretu będącego wynikiem algorytmu ECDH liczony jest skrót przy użyciu algorytmu SHA-256. Następnie jest on dzielony na~dwie części po~128-bitów. Pierwsza część staje się współdzielonym kluczem używanym do~szyfrowania, druga część staje się współdzielonym kluczem używanym do~uwierzytelniania.

Implementacja algorytmu ECDH z~krzywą eliptyczną P-256 pochodzi z~biblioteki \emph{micro-ecc}. Jest to jedyna darmowa biblioteka implementująca ECDH. Wygenerowanie wspólnego sekretu na~mikroprocesorze ATmega32u4 trwa do~4350 ms, nie uwzględniając generowania liczb losowych, używanych przez bibliotekę \emph{micro-ecc} do~zapobiegania atakom typu \emph{side-channel}. Przy uwzględnieniu generowania liczb losowych metodą opisaną w~Dodatku \ref{app:randgen} czas ten rośnie do~4500ms.

\section{Szyfrowanie i deszyfrowanie danych}
\label{sec:encrypt}

Szyfrowanie bloku danych w~wiadomości odbywa się za~pomocą szyfru blokowego AES ze~128-bitowym kluczem używanym w~trybie CBC. Wektor inicjalizacyjny jest losowy i~dołączany do~danych przesyłanej wiadomości przed szyfrogramem. Tekst jawny jest dopełniany do~pełnego bloku według algorytmu zdefiniowanego w~PKCS\#7~\cite{kaliski1998pkcs}.

Stworzona biblioteka nie posiada własnego źródła liczb losowych, musi zostać ono dostarczone w~ramach integracji. Przykładowa metoda generowania liczb losowych na~platformie Arduino została opisana w~Dodatku \ref{app:randgen}.

Właściwe kroki potrzebne do~zaszyfrowania danych wypisano poniżej.

\begin{enumerate}
\item Dopełnienie tekstu jawnego do pełnego bloku:
\begin{itemize}
\item jeżeli długość tekstu jawnego jest wielokrotnością długości bloku, do tekstu jawnego doklejone musi być 16 bajtów o wartości 16,
\item w przeciwnym wypadku, gdy wymagane jest dopełnienie $ N $ bajtów, do tekstu jawnego doklejone musi być $ N $ bajtów o wartości $ N $.
\end{itemize}
\item Zaszyfrowanie dopełnionego tekstu jawnego w trybie CBC z losowym wektorem inicjalizacyjnym.
\item Doklejenie wektora inicjalizacyjnego przed szyfrogramem.
\end{enumerate}

Przykłady dopełniania danych o~różnych długościach przedstawiono w~tabeli~\ref{tab:padding}.

\begin{table}[th]
\centering
\caption{Dopełnanie danych do pełnego bloku. Dopełnienie zaznaczone zostało kolorem niebieskiem i pogrubieniem.}
{\footnotesize Dane ,,Witaj swiecie'' (13 bajtów) zostają dopełnione 3 bajtami o~wartości 3 (0x03):}

\texttt{0x57 0x69 0x74 0x61\\
0x6a 0x20 0x73 0x77\\
0x69 0x65 0x63 0x69\\
0x65 {\color[rgb]{0,0,1}\bfseries 0x03 0x03 0x03}}

{\footnotesize Dane ,,Witaj swiecie !!'' (16 bajtów) zostają dopełnione 16 bajtami o~wartości 16 (0x10):}

\texttt{0x57 0x69 0x74 0x61\\
0x6a 0x20 0x73 0x77\\
0x69 0x65 0x63 0x69\\
0x65 0x20 0x21 0x21\\
{\color[rgb]{0,0,1}\bfseries
0x10 0x10 0x10 0x10\\
0x10 0x10 0x10 0x10\\
0x10 0x10 0x10 0x10\\
0x10 0x10 0x10 0x10\\
}
}

\label{tab:padding}
\end{table}

Kod implementujący szyfrowanie danych przedstawiono w~tabeli~\ref{lst:encrypt} w~dodatku~\ref{app:codesamples}. Zaszyfrowanie 32 bajtów danych na~mikroprocesorze ATmega32u4 trwa do~8 ms bez uwzględnienia czasu potrzebnego do~wygenerowania losowego wektora inicjalizacyjnego. Przy generowaniu losowych danych metodą opisaną w~Dodatku \ref{app:randgen} czas ten rośnie do~64 ms.

Właściwe kroki potrzebne do~odszyfrowania danych wypisano poniżej.

\begin{enumerate}
\item Oddzielenie wektora inicjalizacyjnego od szyfrogramu.
\item Zdeszyfrowanie szyfrogramu w trybie CBC przy wykorzystaniu oddzielonego wektora inicjalizacyjnego.
\item Pobranie wartości ostatniego bajtu zdeszyfrowanego ciągu:
\begin{itemize}
    \item wartość ta nazywana jest dalej $ N $.
\end{itemize}
\item Zweryfikowanie poprawności dopełnienia:
\begin{itemize}
    \item ostatnie $ N $ bajtów musi mieć wartość $ N $,
    \item jeżeli dopełnienie jest nieprawidłowe, cała wiadomość jest ignorowana.
\end{itemize}
\item Usunięcie ostatnich $ N $ bajtów.
\end{enumerate}

Kod implementujący odszyfrowanie danych przedstawiono w~tabeli~\ref{lst:decrypt} w~dodatku~\ref{app:codesamples}. Zdeszyfrowanie 32 bajtów danych na~mikroprocesorze ATmega32u4 trwa do~8 ms.

Implementacja algorytmu AES pochodzi z~biblioteki \emph{AVR-Crypto-Lib}. Jest to najlepiej udokumentowana, darmowa biblioteka implementująca algorytm AES. Implementacja trybu CBC oraz~algorytmu dopełniania zostały zrealizowane w~ramach pracy.

\section{Uwierzytelnienie wiadomości}
\label{sec:auth}

%todo przyklad uwierzytelnienia

Uwierzytelnienie wiadomości odbywa się poprzez~dołączenie MAC do~bloku danych. Dla danej wiadomości MAC tworzony jest za~pomocą szyfru blokowego AES ze~128-bitowym kluczem używanym w~trybie ECBC-MAC. Wektor inicjalizacyjny wypełniony jest zerami i~nie jest przesyłany. Uwierzytelniany jest kompletny szyfrogram wraz z~wektorem inicjalizacyjnym użytym do~szyfrowania, a~nie tekst jawny. Długość szyfrogramu wraz z~wektorem inicjalizacyjnym zawsze będzie wielokrotnością długości bloku, a~więc nie jest stosowane dopełnianie.

Tryb ECBC-MAC to tryb CBC-MAC, którego~wynik jest dodatkowo szyfrowany innym kluczem niż ten użyty do~CBC-MAC. W~tej pracy do~CBC-MAC użyty jest klucz służący do~uwierzytelniania, a~wynik CBC-MAC jest szyfrowany używając klucza służącego do~szyfrowania.

Właściwe kroki potrzebne do~obliczenia kodu uwierzytelniającego opisano poniżej.

\begin{enumerate}
\item Obliczenie ostatniego bloku będącego wynikiem zaszyfrowania szyfrogramu wraz z wektorem inicjalizacyjnym w trybie CBC z wektorem inicjalizacyjnym wypełnionym zerami przy użyciu klucza przeznaczonego do uwierzytelniania.
\item Zaszyfrowanie bloku przy wykorzystaniu AES i klucza przeznaczonego do szyfrowania.
\end{enumerate}

Węzeł wysyłający dokleja kod uwierzytelniający przed szyfrogramem. Węzeł odbierający oddziela otrzymany kod od~szyfrogramu, oblicza kod uwierzytelniający dla danego szyfrogramu i~porównuje, czy~zgadza się on z~kodem otrzymanym. Jeżeli kod obliczony różni się od kodu otrzymanego, cała wiadomość jest ignorowana.

Kod implementujący obliczanie MAC przedstawiono w~tabeli~\ref{lst:mac} w~dodatku~\ref{app:codesamples}. Wygenerowanie kodu uwierzytelniającego dla 32 bajtów danych na~mikroprocesorze ATmega32u4 trwa do~8 ms.

Implementacja trybu ECBC-MAC została zrealizowana w~ramach pracy.

\section{Złożoność implementacji}
\label{cha:complexity}

\FloatBarrier

Stworzona biblioteka implementująca protokół posiada cztery zależności:

\begin{itemize}
    \item moduł \emph{AES} z~biblioteki \emph{AVR-Crypto-Lib},
    \item moduł \emph{gf256mul} wymagany do~obliczeń na~dużych liczbach z~biblioteki \emph{AVR-Crypto-Lib},
    \item moduł \emph{SHA-256} z~biblioteki \emph{AVR-Crypto-Lib},
    \item biblioteka \emph{micro-ecc}.
\end{itemize}

Za~pomocą programu \emph{ctags} obliczono, że~całość biblioteki wraz z~zależnościami składa się ze~131 funkcji oraz~13 struktur. Dla kodu stworzonego w~ramach pracy -- po~odliczeniu zależności -- liczba funkcji to 13 a~struktur to 5. Dodatkowe statystyki dotyczące linii kodu zaprezentowano w~tabelach \ref{tab:stats} oraz~\ref{tab:statssmall}, odpowiednio uwzględniając i~pomijając zależności. Statystyki te wygenerowano przy pomocy programu \emph{cloc}.

\begin{table}[ht]
\centering
\caption{Statystyki złożoności kodu źródłowego}
\begin{subtable}{\textwidth}
    \centering
\begin{tabular}{|l|l|l|l|l|}
    \hline
    \textbf{Typ pliku}  &
    \textbf{Liczba plików}  &
    \textbf{Liczba linii komentarzy}  &
    \textbf{Liczba linii kodu} \\
    \hline
    Nagłówek C/C++ & 24 & 921 & 2466 \\
    \hline
    C & 13 & 402 & 1772 \\
    \hline
    C++ & 3 & 11 & 303 \\
    \hline
    Asembler & 1 & 49 & 39 \\
    \hline
    \hline
    \textbf{Suma} &
    \textbf{41} &
    \textbf{1383} &
    \textbf{4580} \\
    \hline
\end{tabular}
\caption{Statystyki złożoności kodu źródłowego biblioteki wraz z zależnościami.}
\label{tab:stats}
\end{subtable}

\begin{subtable}{\textwidth}
    \centering
\begin{tabular}{|l|l|l|l|}
    \hline
    \textbf{Typ pliku}  &
    \textbf{Liczba plików}  &
    \textbf{Liczba linii komentarzy}  &
    \textbf{Liczba linii kodu} \\
    \hline
    Nagłówek C/C++ & 3 & 83 & 98 \\
    \hline
    C++ & 3 & 11 & 303 \\
    \hline
    \hline
    \textbf{Suma} &
    \textbf{6} &
    \textbf{94} &
    \textbf{401} \\
    \hline
\end{tabular}
\caption{Statystyki złożoności kodu źródłowego biblioteki z wyłączeniem \mbox{zależności}.}
\label{tab:statssmall}
\end{subtable}
\end{table}

