\chapter{Uzyskiwanie liczb losowych}
\label{app:randgen}

\definecolor{mygreen}{rgb}{0,0.6,0}
\definecolor{mygray}{rgb}{0.5,0.5,0.5}
\definecolor{mymauve}{rgb}{0.58,0,0.82}

\lstset{%
  basicstyle=\footnotesize\ttfamily,%
  keywordstyle=\bfseries\color{green!40!black},%
  commentstyle=\itshape\color{purple!40!black},%
  identifierstyle=\color{blue},%
  stringstyle=\color{orange},%
  breakatwhitespace=false,         % sets if automatic breaks should only happen at whitespace
  breaklines=true,                 % sets automatic line breaking
  captionpos=t,                    % sets the caption-position to bottom
  frame=single,	                   % adds a frame around the code
  keepspaces=true,                 % keeps spaces in text, useful for keeping indentation of code (possibly needs columns=flexible)
  language=C,                 % the language of the code
  rulecolor=\color{black}         % if not set, the frame-color may be changed on line-breaks within not-black text (e.g. comments (green here))
}

Biblioteka zaimplementowana w ramach pracy nie posiada żadnego źródła liczb losowych, mimo że jest ono potrzebne do prawidłowego funkcjonowania. Użycie biblioteki wymaga więc dostarczenia źródła liczb losowych przez osobę używającą biblioteki.

Najlepszą metodą jest użycie dedykowanego, zewnętrznego generatora liczb losowych. W bibliotece micro-ecc proponowana jest metoda wykorzystująca szum na niepodłączonym przetworniku analogowo cyfrowym. Jej implementacja zaprezentowana jest w listingu \ref{lst:rand}.

\begin{figure}[!htb]
\begin{lstlisting}[caption={Generowanie liczb losowych w oparciu o wbudowany przetwornik cyfrowo-analogowy. Źródło: biblioteka micro-ecc},label={lst:rand}]
static int RNG(uint8_t *dest, unsigned size) {
  // Use the least-significant bits from the ADC for an unconnected pin (or connected to a source of 
  // random noise). This can take a long time to generate random data if the result of analogRead(0) 
  // doesn't change very frequently.
  while (size) {
    uint8_t val = 0;
    for (unsigned i = 0; i < 8; ++i) {
      int init = analogRead(0);
      int count = 0;
      while (analogRead(0) == init) {
        ++count;
      }
      
      if (count == 0) {
         val = (val << 1) | (init & 0x01);
      } else {
         val = (val << 1) | (count & 0x01);
      }
    }
    *dest = val;
    ++dest;
    --size;
  }
  return 1;
}
\end{lstlisting}
\end{figure}
