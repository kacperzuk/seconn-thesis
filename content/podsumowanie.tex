\chapter*{Podsumowanie}
\addcontentsline{toc}{chapter}{Podsumowanie}
\label{cha:podsumowanie}

Rodzina mikroprocesorów AVR jest często używana do~prototypowania urządzeń, które~przesyłają poufne dane lub~odbierają zdalne komendy, takich jak~bezprzewodowe tokeny, sensory oraz~inteligentne domy. Takie urządzenia powinny być zabezpieczone zarówno przed podsłuchiwaniem informacji jak~i~przed wykonywaniem nieupoważnionych poleceń.

W~pracy przedstawiono rozwiązanie w~postaci protokołu komunikacji oraz~biblioteki programistycznej dla urządzeń AVR, które~ma na~celu zapewnienie poufności, autentyczności oraz~integralności przesyłanych danych. Szczególny nacisk postawiono na~łatwość integracji.

Zaprezentowano różne metody uwierzytelniania i~wybrano algorytmy ECDH (ang. \emph{Elliptic Curve Diffie--Hellman}) oraz~AES użyty w~trybie ECBC-MAC (ang. \emph{Encrypt-last-block Cipher Block Chaining - Message Authentication Code}). Za~ich przewagę nad~innymi rozwiązaniami uznano wydajność, niskie wymagania dotyczące pamięci operacyjnej oraz~jakość implementacji dostępnych na~urządzenia AVR. Dodatkowo w~celu zapewnienia poufności danych wybrano szyfrowanie algorytmem AES użytym w~trybie CBC. Ograniczeniem tej metody jest brak utajnienia przekazywania (ang. \emph{forward secrecy}), a~więc ujawnienie klucza prywatnego dowolnego węzła pozwala na~odszyfrowanie całej przeszłej komunikacji.

Wykorzystano implementacje algorytmów ECDH oraz~AES z~bibliotek \emph{micro-ecc} oraz~\emph{AVR-Crypto-Lib}. W~ramach pracy stworzono implementacje trybów CBC, ECBC-MAC oraz~dopełniania PKCS\#7. Opracowano także protokół służący wymianie kluczy i~ustrukturyzowaniu przesyłanych danych.

Poprawność implementacji poszczególnych algorytmów zweryfikowano poprzez~porównanie zestawów danych wejściowych i~wyjściowych z~niezależnymi implementacjami w~aplikacjach internetowych oraz~w~języku Java na~platformie Android. W~celu potwierdzenia kompletności interfejsu programistycznego biblioteki stworzono przykładowe oprogramowanie na~platformę Arduino, które~oczekuje na~połączenie Bluetooth i~przekazuje uwierzytelnione dane na~połączenie szeregowe. Dodatkowo opracowano implementację protokołu w~języku Java oraz~przykładowe oprogramowanie na~platformę Android, które~nawiązuje połączenie Bluetooth i~pozwala na~szyfrowanie, uwierzytelnianie i~przekazywanie do~drugiego węzła wiadomości. Potwierdzono, że~komunikacja między implementacjami protokołu na~urządzenia AVR oraz~dla języka Java jest prawidłowa.

Przed wdrożeniem rozwiązania powinnien zostać przeprowadzony pełen audyt bezpieczeństwa zarówno protokołu jak~i~jego implementacji. Prawidłowy audyt powinien zostać przeprowadzony przez niezależny zespół, który~nie był powiązany z~projektem protokołu oraz~stworzeniem biblioteki, więc wykracza on poza zakres tej pracy.
